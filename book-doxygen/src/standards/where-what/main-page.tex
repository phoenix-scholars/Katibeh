%
% حق نشر 1390-1402 دانش پژوهان ققنوس
% حقوق این اثر محفوظ است.
% 
% استفاده مجدد از متن و یا نتایج این اثر در هر شکل غیر قانونی است مگر اینکه متن حق
% نشر بالا در ابتدای تمامی مستندهای و یا برنامه‌های به دست آمده از این اثر
% بازنویسی شود. این کار باید برای تمامی مستندها، متنهای تبلیغاتی برنامه‌های
% کاربردی و سایر مواردی که از این اثر به دست می‌آید مندرج شده و در قسمت تقدیر از
% صاحب این اثر نام برده شود.
% 
% نام گروه دانش پژوهان ققنوس ممکن است در محصولات دست آمده شده از این اثر درج
% نشود که در این حالت با مطالبی که در بالا اورده شده در تضاد نیست. برای اطلاع
% بیشتر در مورد حق نشر آدرس زیر مراجعه کنید:
% 
% http://dpq.co.ir/licence
%
\section{برگه نخست}
  برگه نخست ابتدایی ترین قسمت در یک مستند است، از این جهت در ایجاد و سازمان دهی
  یک مستند بسیار مهم است.
  اگر این صفحه به درستی و به شیوه مناسب طراحی و ایجاد نشده باشد، خواننده مستند
  نمی‌تواند به سادگی مستند را مور استفاده قرار دهد و در نخستین برخورد دچاد
  سردرگمی خواهد شد.

  پیش از هر چیز باید به این نکته اشاره کرد که برگه عبارت است از یک بخش از مستند
  که در حالت کلی معادل با یک گفتار در یک کتاب است.
  هر برگه را در یک پرونده \lr{*.doxy} ایجاد می‌شود و در آن متن برگه به صورت کامل
  نوشته می‌شود.
  هر برگه همانند یک گفتار می‌تواند از بخشهای متفاوتی ایجاد شده باشد.
  هر بخش و زیر بخشهای آن نیز به صورت منظم و پشت سرم هم در پرونده برگه نوشته
  می‌شود. در کد زیر نمونه‌ای از یک برگه آورده شده است.
  
\begin{latin}
\lstset{language=C++}
\begin{lstlisting}[frame=single] 
/**
\page pageid Page Title
  Page Body.
  
  \section sectionid Section Title
    Section Body.

    \subsection subsectionid Subsection Title
      Subsection Body.
*/
\end{lstlisting}
\end{latin}

  همانگونه گه پیش از این نیز گفته شد، برگه نخست ابتدایی ترین قسمتی از مستند است
  که هر خواننده با آن روبرو می‌شود. بی شک انتخاب دقیق مطالب مورد نیاز و
  ساختاردهی آن در تاثیر پذیری آن اثر خواهد داشت.
  در نخستین قسمت از برگه نخست باید به معرفی نرم افزار و بیان اهداف اصلی طراحی و
  ایجاد آن اختصاص داده شود.
  هر خواننده با دانستن اهداف یک نرم‌افزار یا بسته نرم‌افزاری می‌تواند به چرایی
  بسیاری از رویکردهای سیستم پاسخ دهد.
  دور از ذهن نیست که برای انجام هر پردازش ویا اجرای هر فرآیند در یک سیستم
  نرم‌افزاری روش‌های متفاوتی وجود داشته باشد. برای نمونه در ذخیره و بازیابی
  داده‌های می‌توان از روش‌های متفاوتی چو پایگاه داده‌ها، پرونده‌های متنی و
  باینری و یا بسیاری از روش‌های دیگر استفاده کرد. واضح است که خواننده با اشراف
  بر اهداف یک سیستم می‌تواند به راحتی درک کند که چرا در یک مسئله از روشی خاص
  استفاده شده است و در نتیجه توانایی و محدودیت سیستم در کجا است.

  در توسعه یک سیستم مدیران پروژه در نخستین گام به دنبال سیستم‌های موجودی خواهند
  بود که بتواند تمام اهداف مورد نظر آنها را پوشش دهد. در بسیاری از موارد نیز
  اهداف و نیازهای سیستم مورد بازبینی قرار گرفته و گه گاه از آنها کاسته می‌شود تا
  کاملا با اهداف یک سیستم موجود هم سو شود. در این صورت ابتدایی ترین پرسش مطرح
  برای یک مدیر اهداف یک سیستم ایجاد شده است. بر این اساس ایجاد یک بخش مجزا و
  تشریح اهداف یک سیستم بسیار اساسی به نظر می‌رسد.

  شکی نیست که پیش از هرچیز در یک مستند تکنیکی باید سیستم را به صورت کامل تعریف
  کرد و اهداف مورد نظر در طراحی و پیاده سازی آن را تشریح کرد اما علاوه بر این
  نیاز است که ساختار مستند نیز به صورت کامل تشریح شود. این تصور اشتبا است که
  همواره هر خواننده از ابتدایی ترین موضوع در مستند شروع کرده و تا انتها تمام
  ستند را به ترتیب مطالعه خواهد کرد.
  خوانندگان یک مستند با دیدگاهی متفاوت تنها به دنبال دسته‌ای خاص از اطلاعات
  ایجاد شده در یک مستند هستند و از خواندن بسیار از بخشهای مستند صرف نظر خواهند
  کرد. در این حالت معرفی مفید و کامل ساختار مستند می‌تواند بسیار مفید واقع شود و
  احساس خوبی را در خوانندگانی که برای اولین بار مستند را مطالعه می‌کنند ایجاد
  کند.

  مدیران  سیستم‌های نرم افزاری در فرآیند گزینش یک سیستم جدید در پی داده‌های چون
  سخت‌افزارهای مورد نیاز، وابستگی به بسته‌های نرم‌افزاری دیگر، قابلیت انتقال به
  روی سکوی‌های متفاوت و دیگر اطلاعات از یک سیستم موجود هستند. این در حالی است که
  توسعه دهندگان سیستم‌های نرم افزاری بیشت به دنبال اطلاعاتی در زمینه به کار گیری
  سیستم در کاربردهای مورد نظر هستند. اینها تنها بخشی از خوانندگان با دیدهای
  متفاوت هستند که مستند تکنیکی سیستم‌های موجود را مورد بررسی و مطالعه قرار
  می‌دهد. در چنین شرایطی ایجاد ساختار مناسب در مستندها و تشریح کافی و کامل آن
  می‌تواند خوانندگان را در یافتن اطلاعات مورد نظرشان از یک سیستم موجود یاری کند.

  سومین بخش از برگه نخست مستند که باید در نظر گرفته شود، بیان سطوح متفاوت موجود
  در یک مستند و توصیه‌های مناسب برای خوانندگان مستند است. بی شک مستند یک سیستم،
  به خصوص زمانی که اندازه سیستم بزرگ است، زمینه‌های متفاوتی را پوشش داده و در
  نتیجه خوانندگان متفاوتی را به خود خواهد دید از این رو تعیین زمینه‌های متفاوت
  موجود و توصیه مناسب به خوانندگانی که قصد مطالعه مستند را دارند می‌تواند آنها
  را در یافتن مناسب اطلاعات مورد نیازشان یاری کند. در کد نمونه‌ای که در ادامه
  آورده شده است یک ساختار ابتدایی برای برگه نخست مستند ارائه شده است.
\begin{latin}
\lstset{language=C++}
\begin{lstlisting}[frame=single] 
/**
\mainpage
  Introduction

\section docstruct Document Structuer
  Document Struction

\subsection whoread Who Read
  Who Read

\section whatsnew New
  What is New

\section bugs Closed Bugs
  Bugs
*/
\end{lstlisting}
\end{latin}

  همان گونه که در متن مستند نوشته شده قابل مشاهد است، دو بخش جدید نیز در انتهای
  برگه نخست پیش بینی شده است.
  اولین بخش در مورد توانایی‌های جدیدی است که به نسخه جدید از سیستم اضافه شده است
  در حالی که دومین بخش در مورد خطا‌هایی است که از نسخه پیشین حذف شده است. این دو
  بخش در رابطه با سیستم‌هایی است که دورده زمانی زیادی را دارند و در نسخه‌های
  متفاوت توسعه می‌یابند.

  