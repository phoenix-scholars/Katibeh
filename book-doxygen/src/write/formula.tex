
\chapter{فرمول‌نویسی}

در مستندات مختلف ممکن است نیاز داشته باشید فرمول‌هایی بنویسید. این وضعیت در
مستندات فنی یا حتی کاربری مربوط به کتابخانه‌های کاربردی بسیار رایج است. فرض کنید
در حال توسعه یک کتابخانه ریاضی هستید و در مستنداتی که برای این کتابخانه
می‌نویسید نیاز به درج روابط و فرمول‌های ریاضی دارید. سوالی که پیش می‌آید این است
که ابزار \lr{Doxygen} چگونه از فرمول‌نویسی پشتیبانی می‌کند؟ به عبارتی چگونه
می‌توان فرمولی را (مثلا در توضیحات یک کلاس یا متد) نوشت تا در مستندات تولید شده
با استفاده از \lr{Doxygen} نیز این فرمول ظاهر شود.
یکی از امکانات مناسب \lr{Doxygen}، این است که برای نوشتن فرمول می‌توان از قوانین
فرمول‌نویسی \lr{\LaTeX} استفاده کرد. در واقع \lr{Doxygen} فرمول‌نویسی بر اساس قوانین
\lr{\LaTeX} را تشخیص می‌دهد و این سبک فرمول‌نویسی را پشتیبانی می‌کند.
با استفاده از این قابلیت می‌توان رابطه‌های بسیار پیچیده ریاضی را در مستندات
نوشت. لازم به ذکر است که استفاده از فرمول‌های \lr{\LaTeX} تنها برای قالب‌های خروجی
ابرمتن (\lr{HTML}) و \lr{\LaTeX} امکان‌پذیر است و برای خروجی‌های در قالب \lr{RTF} و
\lr{man page} کار نمی‌کند.

در این فصل ابتدا توضیح داده می‌شود که فرمول‌هایی که به سبک \lr{\LaTeX} نوشته می‌شوند
در قالب‌های خروجی مختلف چگونه ظاهر می‌شوند و چه ابزاری برای استفاده از این
امکانات نیاز است. سپس فرمول‌نویسی به سبک \lr{\LaTeX} به صورت خلاصه شرح داده می‌شود.
در پایان نیز فرمول‌نویسی مناسب برای \lr{Doxygen} (که بر مبنای قوانین فرمول‌نویسی
\lr{\LaTeX} است) توضیح داده می‌شود.

%
% حق نشر 1390-1402 دانش پژوهان ققنوس
% حقوق این اثر محفوظ است.
% 
% استفاده مجدد از متن و یا نتایج این اثر در هر شکل غیر قانونی است مگر اینکه متن حق
% نشر بالا در ابتدای تمامی مستندهای و یا برنامه‌های به دست آمده از این اثر
% بازنویسی شود. این کار باید برای تمامی مستندها، متنهای تبلیغاتی برنامه‌های
% کاربردی و سایر مواردی که از این اثر به دست می‌آید مندرج شده و در قسمت تقدیر از
% صاحب این اثر نام برده شود.
% 
% نام گروه دانش پژوهان ققنوس ممکن است در محصولات دست آمده شده از این اثر درج
% نشود که در این حالت با مطالبی که در بالا اورده شده در تضاد نیست. برای اطلاع
% بیشتر در مورد حق نشر آدرس زیر مراجعه کنید:
% 
% http://dpq.co.ir/licence
%
% MINDSTORME : مصطفی ۱۳۹۰-۱۲ : بررسی قالب‌های خروجی بررسی قالب‌های خروجی در
% فرمول نویسی به صورت بخش‌های جداگانه می‌تواند مفید باشد.
% البته می‌توان این کار را در قسمتی جداگانه نیز انجام داد - برای هر قالب یک فصل
% NOTE: هادی ۵-۱۳۹۱: بررسی قالب خروجی (جواب) توضیحات مربوط به قالب های خروجی
% مختلف برای فرمول نویسی اونقدر نیست که بشه در یک فصل نوشت. بنابراین فعلا قرار
% دادن تمام اونها در یک بخش از فصل فرمول‌نویسی منطقی تر به نظر می رسه.

\section{قالب‌های خروجی}
اینکه \lr{Doxygen} فرمول‌های نوشته شده در یک پروژه یا پرونده را چگونه در مستندات
نهایی تولید شده قرار می‌دهد به قالب مستند نهایی بستگی دارد (به اینکه شما بخواهید
مستند نهایی را در قالب \lr{\LaTeX}، ابرمتن (یا \lr{HTML})، \lr{RTF} و یا قالبی
دیگر تولید کنید).

قالب های زیر بطور مستقیم توسط \lr{Doxygen} پشتیبانی می شوند. یعنی اینکه اگر قالب
خروجی مورد نظر برای مستند نهایی یکی از موارد زیر باشد می‌توان از فرمول‌نویسی به
سبک \lr{\LaTeX} در مستندات منبع استفاده کرد و در نهایت با ابزار \lr{Doxygen}
مستند نهایی را تولید کرد. این قالب‌ها عبارتند از:

\begin{itemize}
	\item \lr{HTML}
	\item \lr{Latex}
\end{itemize}

برای تولید مستندات در قالب ابرمتن باید در پرونده پیکربندی تگ \lr{GENERATE\_HTML}
فعال شود (با مقدار \lr{YES} مقداردهی شود) و برای تولید مستندات در قالب
\lr{\LaTeX} باید تگ \lr{GENERATE\_LATEX} فعال شود.

در فرآیند تولید مستند فنی، زمانی که قالب خروجی به صورت \lr{\LaTeX} تنظیم
شده باشد، تنها کافی است که فرمول‌های نوشته شده در مستندات منبع با تغییراتی
جزیی در مستندات نهایی خروجی رونوشت شود و این دقیقا همان کاری است که
\lr{Doxygen} برای تولید مستندات در این قالب انجام می‌دهد. اما اگر بخواهیم مستند
تولید شده نهایی در قالبی غیر از قالب  \lr{\LaTeX} باشد در این صورت لازم است
روابط و فرمول‌های نوشته شده در مستندات منبع به طریقی ترجمه شوند که برای قالب
مستند نهایی تولید شده قابل تشخیص و نمایش باشد (به عنوان مثال فرمول‌ها به عکس
تبدیل شوند).

\begin{note}
در حال حاضر برای تولید مستندات نهایی در قالب‌های \lr{RTF} و یا \lr{man page}
امکان ترجمه رابطه‌ها و فرمول وجود ندارد که به نوبه خود یک محدودیت اساسی در این ابزار به شمار می‌آید.
\end{note}

یکی از پرکاربردترین قالب‌های خروجی قالب ابرمتن یا \lr{HTML} است. خوشبختانه
امکان تبدیل فرمول‌های نوشته شده در مستندات منبع به گونه‌ای که در قالب ابرمتن
قابل تشخیص و نمایش باشد وجود دارد. برای اینکه \lr{Doxygen} بتواند فرمول‌ها را به
صورت تصویر در مستند نهایی ابرمتنی قرار دهد نیاز به چند ابزار جانبی دارد تا با
استفاده از آن‌ها رابطه‌ها و فرمول‌های موجود در مستندات منبع را به تصویر تبدیل
کند و سپس آن‌ها را در مستندات نهایی قرار دهد. ابزارهای جانبی‌ای که باید روی
سیستم نصب شده باشد عبارتند از:

\begin{description}
 \item [\lr{latex}]:
	مترجم دستورات \lr{\LaTeX} که برای تجزیه و ترجمه روابط و فرمول‌ها مورد نیاز است. 
 \item [\lr{dvips}]:
 	ابزاری برای تبدیل پرونده‌های از نوع \lr{DVI} به نوع \lr{Post Script}.
  \item [\lr{gs}]:
  مفسر \lr{Ghost Script} مفسری است که با استفاده از آن می‌توان پرونده‌هایی از
  نوع \lr{Post Script} را به عکس‌هایی در قالب \lr{bitmap} تبدیل کرد.
\end{description}

% بسته نرم‌افزاری \lr{JaTex} یک مترجم است که در ترجمه رابطه‌های ریاضی مورد استفاده
% قرار می‌گیرد.
% بسته \lr{dvips} نیز برای تبدیل پرونده‌های \lr{DVI}به \lr{Post Script} مورد
% استفاده قرار می‌گیرد.
روابط و فرمول‌های نوشته شده در مستند منبع با استفاده از مترجم \lr{\LaTeX} به
پرونده‌هایی از نوع \lr{DVI} تبدیل می‌شوند. سپس این پرونده‌ها با استفاده از
\lr{dvips} به پرونده‌هایی در قالب \lr{Post Script} تبدیل می‌شوند و این پرونده‌ها
در نهایت با استفاده از مفسر \lr{gs} به تصاویری از نوع \lr{bitmap} تبدیل
می‌شوند که حاوی فرمول مورد نظر هستند.

%  پرونده‌های \lr{DVI} با استفاده از مترجم \lr{LaTex}
% از رابطه‌های ریاضی نوشته شده در مستند ایجاد می‌شود. در نهایت با استفاده بسته \lr{Gost Script} تصاویر قابل
% نمایش در خروجی ایجاد خواهد شد.
% ذکر این نکته لازم است که، برای استفاده از این قالب خروجی نصب بودن مترجم
% \lr{LaTex} الزامی است.

برای مستندات در قالب ابرمتن یک روش دیگر هم وجود دارد که در این روش فرمول‌ها
و روابط ریاضی موجود در مستندات منبع به عکس تبدیل نمی‌شوند بلکه خود فرمول‌ها
در مستند نهایی قرار می‌گیرند. با این روش دیگر نیازی به ابزارهای فوق نیست (چون
ابزارهای فوق برای تبدیل فرمول‌ها به عکس مورد نیازند). در این روش از \lr{MathJax}
استفاده می‌شود. برای استفاده از این روش باید در پرونده پیکربندی \lr{Doxygen} تگ
\lr{USE\_MATHJAX} را فعال کنید (این تگ را \lr{YES} مقداردهی کنید). با این کار
فرمول‌های نوشته شده به سبک \lr{\LaTeX} در مستند منبع، به همان صورت در مستند
تولید شده نهایی (در قالب ابرمتن) نوشته می‌شوند. در هنگام نمایش این مستندات نهایی
(که طبعا با استفاده از یک مرورگر نمایش می‌یابد) برنامه \lr{MathJax} که
یک برنامه جاوا-اسکریپت است فرمول‌های موجود را به تصویر تبدیل کرده و
نمایش می‌دهد. بنابراین برای تولید مستندات ابرمتن به این روش نیازی به نصب ابزار خاصی
نیست و کاربری که می‌خواهد مستندات را مشاهده کند باید روی سیستم خود ابزار
\lr{MathJax} را نصب کرده باشد تا مرورگر بتواند با استفاده از آن فرمول‌های موجود
در مستندات را نمایش دهد. نکته اینکه ابزار \lr{MathJax} نرم‌افزاری متن‌باز و
رایگان است\footnote{تارنمای رسمی این نرم‌افزار \lr{http://www.mathjax.com}
است.}.

% NOTE: هادی ۵-۱۳۹۱: در این قسمت در موردنحوه ظهور فرمول ها در  قالب‌های مختلف خروجی صحبت می‌شود.
% هادی ۶-۱۳۹۱: در این قسمت باید در مورد نحوه فرمول نویسی در لاتک توضیح داده شود.
% تا حد امکان خلاصه و جامع. علاوه بر آن در مورد محیط‌های مختلفی که برای
% فرمول‌نویسی در لاتک وجود دارند نیز باید صحبت شود و کاربرد و نیازهای این محیط‌ها
% شرح داده شود.
\section{فرمول نویسی به سبک \lr{\LaTeX}}
\label{sec:latex-formula}
در این قسمت فرمول‌نویسی به سبک \lr{\LaTeX} شرح داده می‌شود. باید به این نکته
توجه داشت که برای ترجمه متون نوشته شده به زبان \lr{\LaTeX} و فرمول‌هایی که در آن
به کار رفته ممکن است نیاز به این باشد که بسته‌های خاصی را در مستند نوشته شده به
زبان \lr{\LaTeX} وارد کنید. ما در اینجا در مورد اینکه چگونه به زبان \lr{\LaTeX}
مستند نویسی کنید و یا اینکه از چه بسته‌هایی استفاده کنید صحبت نخواهیم کرد بلکه
در این قسمت تنها به این موضوع می‌پردازیم که فرمول‌نویسی به سبک \lr{\LaTeX} چگونه
است. این سبک فرمول‌نویسی را می‌توانید در مستندات برنامه‌نویسی خود (لابه‌لای
کدها در جاهایی که نیاز به نوشتن فرمول دارید) بنویسید و سپس با استفاده از ابزار
\lr{Doxygen} مستند فنی خود را در قالب ابرمتن و یا در قالب \lr{\LaTeX} (به صورتی
که در قسمت قبل شرح داده شد) تولید کنید. تاکید می‌کنیم که ما در این قسمت و به طور
کلی در این کتاب قصد نداریم مستند نویسی به زبان \lr{\LaTeX} را شرح دهیم و همچنین
در مورد اینکه بعد از تولید مستندات فنی در قالب \lr{\LaTeX} توسط \lr{Doxygen}
چگونه این مستند را به قالب‌های دیگر تبدیل کنید نیز صحبت نخواهیم کرد. فقط به همین
نکته بسنده می‌کنیم که اغلب برای نوشتن فرمول‌های \lr{\LaTeX} از بسته
\lr{amsmath}  یا \lr{mathtools} استفاده می‌شود. یعنی از دستور \lr{\textbackslash
usepackage\{amsmath\}} یا \lr{\textbackslash
usepackage\{amsmath\}} در ابتدای مستند به زبان \lr{\LaTeX} استفاده می‌شود. بسته
\lr{amsmath} توسط انجمن ریاضی آمریکا\LTRfootnote{AMS: American Mathematical
Society} منتشر شده است.

برای فرمول‌نویسی اولین کاری که باید انجام شود این است که به طریقی اعلام کنیم که
می‌خواهیم فرمول بنویسیم! به همین دلیل در \lr{\LaTeX} محیط‌هایی تعریف شده است که
فرمول باید در یکی از این محیط‌ها نوشته شود. در شکل کلی یک فرمول به این صورت
نوشته می‌شود:

\begin{latex}
\begin{environment}
formula
\end{environment}
\end{latex}

که در آن به جای کلمه \lr{environmet} نام محیط مناسب مورد نظر و به جای کلمه
\lr{formula} فرمول مورد نظر (با قواعدی که در ادامه خواهد آمد) نوشته می‌شود.

به طور کلی فرمول‌ها را می‌توانیم به دو دسته تقسیم کنیم:

\begin{itemize}
  \item بین متن: فرمول‌هایی که در جمله متنی ظاهر می‌شوند.
  \item جداگانه: فرمول‌هایی که به طور جداگانه و خارج از خطوط و جملات متن قرار
  می‌گیرند (مثلا در یک پاراگراف جدا).
\end{itemize}

برای نوشتن فرمول‌های بین متنی از محیط \lr{math} استفاده می‌شود و برای فرمول‌های
جداگانه معمولا از محیط \lr{displaymath} یا \lr{equation*} استفاده می‌شود.
نوشتن فرمول در این محیط‌ها باعث می‌شود فرمول در خطی جداگانه و البته درست وسط
خط قرار داده شود. به مثال‌های ساده زیر توجه کنید:

\begin{latex}
The simple formula,
\begin{math}
a+a=2a
\end{math}
, is a text formula that write
in a text line. But the below formula is a seperate formula:
\begin{displaymath}
a+a=2a
\end{displaymath}
This formula can write as bellow:
\begin{equation*}
a+a=2a
\end{equation*}
These formuls are in a seperate line and in the center of line.
\end{latex}

نتیجه این کد \lr{\LaTeX} به صورت زیر خواهد بود:

\begin{latin}
The simple formula,
\begin{math}
a+a=2a
\end{math}
, is a text formula that write
in a text line. But the below formula is a seperate formula:
\begin{displaymath}
a+a=2a
\end{displaymath}
This formula can write as bellow:
\begin{equation*}
a+a=2a
\end{equation*}
These formuls are in a seperate line and in the center of line.
\end{latin}

محیط‌های دیگری هم برای فرمول‌نویسی در \lr{\LaTeX} وجود دارند که هر یک ویژگی‌های
خاصی دارند و در همین بخش به آن‌ها اشاره خواهیم کرد اما محیط‌های فوق بسیار متداول
هستند به همین دلیل کوتاه‌نویسی‌هایی برای این محیط‌ها در \lr{\LaTeX} وجود دارد.
به این ترتیب که به جای تعیین ابتدا و انتهای محیط \lr{math} می‌توان از دو علامت
$\$$ استفاده کرد. یعنی اینکه برای نوشتن فرمول‌ها در یک خط می‌توان فرمول را بین
دو علامت $\$$ قرار داد. به جای محیط‌های \lr{displaymath} و \lr{equation*} نیز
می‌توان از \lr{\textbackslash [} و \lr{\textbackslash ]} استفاده کرد. به مثال
زیر توجه کنید:

\begin{latex}
The simple formula, $a+a=2a$, is a text formula that write
in a text line. But the below formula is a seperate formula:
\[
a+a=2a
\]
This formula can write as bellow:
\[
a+a=2a
\]
These formuls are in a seperate line and in the center of line.
\end{latex}

نتیجه این کد \lr{\LaTeX} به صورت زیر خواهد بود. مشاهده می‌شود که خروجی مثال زیر
دقیقا مثل خروجی مثال قبل است.

\begin{latin}
The simple formula, $a+a=2a$, is a text formula that write
in a text line. But the below formula is a seperate formula:
\[
a+a=2a
\]
This formula can write as bellow:
\[
a+a=2a
\]
These formuls are in a seperate line and in the center of line.
\end{latin}

فعلا همین دو محیط را در خاطر داشته باشید. در ادامه به طور خلاصه نحوه نوشتن
نمادها و علامت‌های مختلف در فرمول‌‌های \lr{\LaTeX} و قواعد مربوط به فرمول‌نویسی
راشرح می‌دهیم و پس از آن در قسمت \ref{sec:formula-environments} محیط‌های دیگر را
معرفی می‌کنیم.

% TODO: Hadi - 1391: زیربخش های زیر تکمیل شود تا این فصل به اتمام برسد.
% Symbols
\subsection{نمادهای ریاضی}
در ریاضیات نمادها و نشانه‌های بسیار زیادی وجود دارد و برای وارد کردن هر یک از
نمادها در فرمول مجبورید دستور مربوط به آن نماد را وارد کنید. برای برخی نمادهای
متداول، همان حرف مربوط به آن نماد کافی است. این دسته از نمادها را در زیر
می‌بینید:
\begin{latex}
+ - = ! / ( ) [ ] < > | ' :
\end{latex} 
اما برای نمادهای دیگر (مثل حروف یونانی، عملگرهای مجموعه‌ای، رابطه‌ای، منطقی و
غیره) باید دستور خاص آن نماد را بنویسید که البته تعداد این نمادها و در نتیجه
دستورهایی که باید به خاطر سپرد بسیار زیاد است. در قسمت‌های بعدی دستورات مربوط به
نمادهایی چون حروف یونانی، عملگرها، توان، انتگرال، جذر و غیره که پرکاربرد هستند
شرح داده می‌شوند. اگر نماد مورد نظر خود را در این قسمت از این کتاب نیافتید
مجبورید جستجو کرده و دستور مربوط به آن نماد را بیابید. خوشبختانه روش‌هایی وجود
دارد که جستجوی دستور مربوط به نمادها را ساده می‌کند. یک راه سریع، جالب و ساده
مراجعه به تارنمای \lr{Detexify} است\LTRfootnote{http://detexify.kirelabs.org/}. در این تارنما محلی
وجود دارد که کافی است برای یافتن دستور مربوط به نماد مورد نظر خود، آن نماد را در
قسمت تعیین شده ترسیم کنید! راه دیگر مراجعه به فهرست جامع نمادهای
\lr{\LaTeX}
است\LTRfootnote{http://www.ctan.org/tex-archive/info/symbols/comprehensive/symbols-letter.pdf}.
در ادامه نحوه وارد کردن برخی نمادها و نشانه‌ها بیان می‌شود.

% Greek Letters
\subsection{حروف یونانی}
استفاده از حروف یونانی در فرمول‌ها و روابط ریاضی بسیار رایج است. برای وارد کردن
یک حرف یونانی کافی است نام آن حرف یونانی (مثلا \lr{sigma}، \lr{delta} و
\lr{alpha}) را بعد از یک علامت \lr{\textbackslash} بنویسید. اگر حرف اول نام را
با حرف کوچک\LTRfootnote{Lowercase}  بنویسید حرف یونانی به صورت حرف کوچک ظاهر
می‌شود. برای وارد کردن حرف یونانی به صورت حرف بزرگ\LTRfootnote{Uppercase} کافی
است هنگام نوشتن نام آن حرف یونانی بعد از علامت \lr{\textbackslash} حرف اول نام
را به صورت بزرگ بنویسید. در زیر مثالی از وارد کردن چند حرف یونانی نشان داده شده است:

\begin{latex}
\[
\alpha, \beta, \gamma, \Gamma, \pi, \Pi, \phi, \varphi, \Phi
\]
\end{latex}
\[
\alpha, \beta, \gamma, \Gamma, \pi, \Pi, \phi, \varphi, \Phi
\]

\begin{note}
حروف یونانی پی ($\pi$، $\varpi$)، اپسیلون ($\epsilon$، $\varepsilon$)، تتا
($\theta$، $\vartheta$)، فی ($\phi$، $\varphi$)، رو ($\rho$، $\varrho$) و سیگما
($\sigma$، $\varsigma$) به دو صورت نوشته می‌شوند. برای وارد کردن شکل دوم این
حروف کافی است قبل از نام (یعنی بعد از علامت \lr{\textbackslash} و قبل از نام)
کلمه \lr{var} نوشته شود (جدول \ref{greek-letters-table} را ببینید).
\end{note}

\begin{note}
برخی حروف یونانی در حالت بزرگ درست شبیه حروف لاتین هستند. به عنوان مثال حروف
یونانی آلفا و بتا در حالت بزرگ به صورت \lr{A} و \lr{B} نمایش می‌یابند.
به همین دلیل در \lr{\LaTeX} برای این حروف، شکل بزرگ تعریف نشده و برای استفاده از
آن‌ها باید از همان حروف لاتین استفاده کنید. در جدول \ref{greek-letters-table}
فهرستی از حروف یونانی آورده شده است. در این جدول خانه‌هایی که شکل متناظر برای
آن‌ها وجود ندارد همان حروفی هستند که شکل متناظر آن‌ها مثل همان حروف لاتین هستند.
بنابراین نباید از دستور آن‌ها استفاده شود. مثلا برای نوشتن حرف آلفا در حالت بزرگ
باید از حرف لاتین \lr{A} استفاده کرد.
\end{note}


در جدول \ref{greek-letters-table} فهرست تقریبا کاملی از حروف یونانی و دستورات لازم برای وارد کردن
آن‌ها آورده شده است.

\begin{table}
\begin{latin}
\centering
\begin{tabular}{|l|c||l|c||l|c|}
\hline
\rl{دستور}					&	\rl{نمایش}	&	\rl{دستور}					&	\rl{نمایش}	&	\rl{دستور}					&	\rl{نمایش}	\\ \hline\hline
\textbackslash Alpha 		&		 		&	\textbackslash Iota 		&				&	\textbackslash rho			&	$\rho$		\\ \hline
\textbackslash alpha		&	$\alpha$	&	\textbackslash iota 		& 	$\iota$		&	\textbackslash varrho 		&   $\varrho$	\\ \hline
\textbackslash Beta 		&  			 	&	\textbackslash Kappa  		& 				& 	\textbackslash Sigma 		&	$\Sigma$	\\ \hline
\textbackslash beta			&	$\beta$		&	\textbackslash kappa  		&	$\kappa$	&	\textbackslash sigma 		&	$\sigma$	\\ \hline
\textbackslash Gamma		&	$\Gamma$	&	\textbackslash Lambda  		&	$\Lambda$ 	&	\textbackslash varsigma  	& 	$\varsigma$	\\ \hline
\textbackslash gamma		&	$\gamma$	&	\textbackslash lambda 		&	$\lambda$	&	\textbackslash Tau 			&				\\ \hline
\textbackslash Delta 		&	$\Delta$	&	\textbackslash Mu 			&				&	\textbackslash tau 			&	$\tau$		\\ \hline
\textbackslash delta		&	$\delta$	&	\textbackslash mu 			&	$\mu$		&	\textbackslash Upsilon 		&	$\Upsilon$	\\ \hline
\textbackslash Epsilon		&				&	\textbackslash Nu 			&				& 	\textbackslash upsilon 		&	$\upsilon$	\\ \hline
\textbackslash epsilon		&	$\epsilon$	&	\textbackslash nu 			& 	$\nu$		&	\textbackslash Phi 			&	$\Phi$		\\ \hline
\textbackslash varepsilon	&	$\varepsilon$&	\textbackslash Xi  			&	$\Xi$		&	\textbackslash phi 			&	$\phi$		\\ \hline
\textbackslash Zeta			&				&	\textbackslash xi 			& 	$\xi$		&	\textbackslash varphi 		&	$\varphi$	\\ \hline
\textbackslash zeta			&	$\zeta$		&	\textbackslash Omicron  	&				&	\textbackslash Chi 			&				\\ \hline
\textbackslash Eta			&				&	\textbackslash omicron  	&				&	\textbackslash chi 			&	$\chi$		\\ \hline
\textbackslash eta			&	$\eta$		&	\textbackslash Pi 			&	$\Pi$		&	\textbackslash Psi 			&	$\Psi$		\\ \hline
\textbackslash Theta		&	$\Theta$	&	\textbackslash pi 			&	$\pi$		&	\textbackslash psi 			&	$\psi$		\\ \hline	
\textbackslash theta		&	$\theta$	&	\textbackslash varpi 		&	$\varpi$	&	\textbackslash Omega 		&	$\Omega$	\\ \hline
\textbackslash vartheta		&	$\vartheta$	&	\textbackslash Rho 			&				&	\textbackslash omega		&	$\omega$	\\ \hline
\end{tabular}
\end{latin}
\caption{حروف یونانی به همراه دستور مربوط به حروف}
\label{greek-letters-table}
\end{table}

% Powers and indices
\subsection{بالانویس و پایین‌نویس (توان و اندیس)}
توان و اندیس در روابط ریاضی متناظر بالانویس و پایین‌نویس در متن‌های معمولی است.
برای توان از علامت شاپو ( \lr{$\^$} ) و برای اندیس نیز از علامت
زیرخط\LTRfootnote{Underscore} (\lr{\_}) استفاده می‌شود. نکته اینکه در صورتی که
بخواهید چند کاراکتر (مثلا یک عبارت یا کلمه) را به عنوان اندیس یا توان استفاده
کنید باید تمام آن‌ها را بعد از علامت اندیس یا توان بین دو آکولاد (یعنی \{ و\})
قرار دهید. در زیر مثال‌هایی از اندیس و توان مشاهده می‌شود:

\begin{latex}
\[
F_{n+1} = k^2 + F_n^2 - F_{n-1} + k^{n^2+n+1}_n + F^{n_3}
\]
\end{latex}

\[
 F_{n+1} = k^2 + F_n^2 - F_{n-1} + k^{n^2+n+1}_n + F^{n_3}
\]

% Operators
\subsection{عملگرها}
تعداد بسیار زیادی عملگر یا همان عمل ریاضی وجود دارد که در اینجا به مهم‌ترین و
پرکاربردترین آن‌‌ها اشاره می‌شود. عملگرها را به دسته‌های محاسباتی و منطقی،
رابطه‌ای، پیکانی، مجموعه‌ای و متفرقه تقسیم کرده‌ایم و در ادامه دستور وارد کردن
عملگرهای هر یک از این دسته‌ها را شرح می‌دهیم. توجه شود که اغلب این عملگرها دو
عملوندی هستند. بنابراین باید در طرف راست و چپ آن‌ها عباراتی به عنوان عملوند قرار
داده شود.
\subsubsection{محاسباتی و منطقی}
مهم‌ترین عملگرهای ریاضی همین دسته است که شامل چهار عمل اصلی «جمع»، «ضرب»،
«تفریق»، «تقسیم» و عمل‌های منطقی «یا\LTRfootnote{OR}»، «و\LTRfootnote{AND}»،
«نقیض\LTRfootnote{NOT}»، «یای انحصاری\LTRfootnote{XOR}» و عملگرهای بسیار دیگر
است. در جدول \ref{math-logic-operators-table} این علائم و دستور مربوط به هر یک برای وارد کردن در فرمول‌های
\lr{\LaTeX} آورده شده است.

\begin{table}
\begin{latin}
\centering
\begin{tabular}{|l|c||l|c||l|c||l|c||l|c|}
\hline
\rl{دستور}				&	\rl{نمایش}	&	\rl{دستور}					&	\rl{نمایش}	&	\rl{دستور}					&	\rl{نمایش}	&	\rl{دستور}						&	\rl{نمایش}		&	\rl{دستور}				&	\rl{نمایش}		\\ \hline\hline
+					 	&	+	 		&	\textbackslash boxdot 		&	$\boxdot$	&	\textbackslash dotplus		&	$\dotplus$	&	\textbackslash bigtriangledown	&	$\bigtriangledown$&	\textbackslash bigcap	&	$\bigcap$		\\ \hline
- 						&  	-			&	\textbackslash boxminus  	& 	$\boxminus$	& 	\textbackslash mp	 		&	$\mp$		&	\textbackslash bigtriangleup	&	$\bigtriangleup$&	\textbackslash bigcup	&	$\bigcup$		\\ \hline
/						&	/			&	\textbackslash boxplus  	&	$\boxplus$	&	\textbackslash pm 			&	$\pm$		&	\textbackslash triangleleft		&	$\triangleleft$	&	\textbackslash bigodot	&	$\bigodot$		\\ \hline
\textbackslash div		&	$\div$		&	\textbackslash boxtimes 	&	$\boxtimes$	&	\textbackslash land \rl{یا} \textbackslash wedge 		&	$\land$		&	\textbackslash triangleright	&	$\triangleright$	&	\textbackslash bigoplus	&	$\bigoplus$	\\ \hline
*						&	*			&	\textbackslash odot 		& 	$\odot$		&	\textbackslash barwedge		&   $\barwedge$	&	\textbackslash cap				&	$\cap$			&	\textbackslash bigotimes&	$\bigotimes$	\\ \hline
\textbackslash times	&	$\times$	&	\textbackslash ominus  		&	$\ominus$ 	&	\textbackslash land \rl{یا} \textbackslash vee  		& 	$\lor$		&	\textbackslash cup		&	$\cup$&	\textbackslash bigsqcup	&	$\bigsqcup$	\\ \hline
\textbackslash ltimes 	&	$\ltimes$	&	\textbackslash oplus		&	$\oplus$	&	\textbackslash veebar 		&	$\veebar$	&	\textbackslash sqcap			&	$\sqcap$		&	\textbackslash bigvee	&	$\bigvee$		\\ \hline
\textbackslash rtimes	&	$\rtimes$	&	\textbackslash otimes		&	$\otimes$	&	\textbackslash dagger 		&	$\dagger$	&	\textbackslash sqcup			&	$\sqcup$		&	\textbackslash bigwedge	&	$\bigwedge$		\\ \hline
\textbackslash star		&	$\star$		&	\textbackslash oslash		&	$\oslash$	& 	\textbackslash ddagger 		&	$\ddagger$	&	\textbackslash bmod				&	$\bmod$			&							&					\\ \hline
\end{tabular}
\end{latin}
\caption{عملگرهای محاسباتی و منطقی به همراه دستور مربوط به عملگرها}
\label{math-logic-operators-table}
\end{table}

\begin{note}
برای عملگر پیمانه\LTRfootnote{Madular operator} (همنهشتی) دو دستور
(\lr{\textbackslash bmod} و \lr{\textbackslash pmod}) وجود دارد.
در مثال زیر این دو دستور و نحوه نمایش آن‌ها مشاهده می‌شود.
\end{note}
\begin{latex}
\[
 a \bmod b
\]
\[
x \equiv a \pmod b
\]
\end{latex}
\[
 a \bmod b
\]
\[
x \equiv a \pmod b
\]

\subsubsection{رابطه‌ای}
عملگرهای رابطه‌ای شامل «بزرگتر»، «کوچکتر»، «مساوی» و مشتقات این عملگرهاست. در
جدول \ref{relational-operators-table} عملگرهای پرکاربرد و مهم این دسته آورده شده است.

\begin{table}
\begin{latin}
\centering
\begin{tabular}{|l|c||l|c||l|c||l|c|}
\hline
% TODO: Hadi - 6-1391: علامت‌ها در جدول بر اساس ترتیبی خاص مرتب شود. از نظر
% ستونی مرتب هست: ستون اول علامتهای بزرگتری، ستون دوم کوچکتری و ستون سوم مساوی
\rl{دستور}					&	\rl{نمایش}	&	\rl{دستور}					&	\rl{نمایش}	&	\rl{دستور}					&	\rl{نمایش}	&	\rl{دستور}					&	\rl{نمایش}	\\ \hline\hline
<					 		&	<	 		&	>							&	>			&	= 							&  	=			&	\textbackslash triangleq 	&	$\triangleq$\\ \hline
\textbackslash leq  \rl{یا} \textbackslash le	&	$\leq$		&	\textbackslash geq  \rl{یا} \textbackslash ge			& 	$\geq$		& 	\textbackslash neq \rl{یا} \textbackslash ne 	&	$\neq$		&	\textbackslash circeq		&	$\circeq$		\\ \hline
\textbackslash leqq			&	$\leqq$		&	\textbackslash geqq  		&	$\geqq$ 	&	\textbackslash cong			&	$\cong$		&	\textbackslash propto		&	$\propto$	\\ \hline
\textbackslash leqslant		&	$\leqslant$	&	\textbackslash geqslant		&	$\geqslant$	& 	\textbackslash ncong 		&	$\ncong$	&	\textbackslash varpropto	&	$\varpropto$\\ \hline
\textbackslash ll	 		&	$\ll$		&	\textbackslash gg			&	$\gg$		&	\textbackslash sim			&	$\sim$		&								&				\\ \hline
\textbackslash lll  \rl{یا} \textbackslash llless			&   $\lll$		&	\textbackslash ggg  \rl{یا} \textbackslash gggtr			&	$\ggg$		&	\textbackslash nsim			&	$\nsim$		&								&				\\ \hline
\textbackslash lneq  		& 	$\lneq$		&	\textbackslash gneq			&	$\gneq$		&	\textbackslash simeq		&	$\simeq$	&								&				\\ \hline
\textbackslash lneqq 		&	$\lneqq$	&	\textbackslash gneqq		&	$\gneqq$	&	\textbackslash backsim		&	$\backsim$	&								&				\\ \hline
\textbackslash lesseqgtr 	&	$\lesseqgtr$&	\textbackslash gtreqless	&	$\gtreqless$&	\textbackslash backsimeq	&	$\backsimeq$&								&				\\ \hline
\textbackslash lesseqqgtr 	&	$\lesseqqgtr$&	\textbackslash gtreqqless	&	$\gtreqqless$&	\textbackslash approx		&	$\approx$	&								&				\\ \hline
\textbackslash lessgtr 		&	$\lessgtr$	&	\textbackslash gtrless		&	$\gtrless$	&	\textbackslash approxeq		&	$\approxeq$	&								&				\\ \hline
\textbackslash lvertneqq	&	$\lvertneqq$&	\textbackslash gvertneqq	&	$\gvertneqq$&	\textbackslash thickapprox	&	$\thickapprox$&								&				\\ \hline
\textbackslash nleq			&	$\nleq$		& 	\textbackslash ngeq 		&	$\ngeq$		&	\textbackslash thicksim  	& 	$\thicksim$	&								&				\\ \hline
\textbackslash nleqq		&	$\nleqq$	& 	\textbackslash ngeqq 		&	$\ngeqq$	&	\textbackslash equiv		&	$\equiv$	&								&				\\ \hline
\textbackslash nleqslant	&	$\nleqslant$&	\textbackslash ngeqslant 	&	$\ngeqslant$&	\textbackslash doteq		&	$\doteq$	&								&				\\ \hline
\textbackslash nless		&	$\nless$	&	\textbackslash ngtr			&	$\ngtr$		&	\textbackslash doteqdot  \rl{یا} \textbackslash Doteq		&	$\doteqdot$	&								&				\\ \hline
\end{tabular}
\end{latin}
\caption{عملگرهای رابطه‌ای به همراه دستور مربوط به عملگرها}
\label{relational-operators-table}
\end{table}

\subsubsection{پیکانی}
این دسته از عملگرها که اغلب عملگرهایی با دو عملوند هستند عملگرهایی مثل «نتیجه
می‌دهد»، «استنتاج»، «اگر و تنها اگر» و غیره است. در جدول \ref{arrow-operators-table} عملگرهای
پرکاربرد از این مجموعه آورده شده است.

\begin{table}
\begin{latin}
\centering
\begin{tabular}{|l|c||l|c||l|c||l|c|}
\hline
\rl{دستور}						&	\rl{نمایش}			&	\rl{دستور}						&	\rl{نمایش}			&	\rl{دستور}							&	\rl{نمایش}			&	\rl{دستور}						&	\rl{نمایش}			\\ \hline\hline
\textbackslash circlearrowleft	&	$\circlearrowleft$	&	\textbackslash downdownarrows 	&	$\downdownarrows$	&	\textbackslash Leftarrow			&	$\Leftarrow$			&	\textbackslash nLeftarrow		&	$\nLeftarrow$	\\ \hline
\textbackslash circlearrowright	&  	$\circlearrowright$	&	\textbackslash upuparrows  		& 	$\upuparrows$		& 	\textbackslash Longleftarrow		&	$\Longleftarrow$		&	\textbackslash nRightarrow		&	$\nRightarrow$	\\ \hline
\textbackslash curvearrowleft	&	$\curvearrowleft$	&	\textbackslash leftleftarrows  	&	$\leftleftarrows$	&	\textbackslash Rightarrow			&	$\Rightarrow$			&	\textbackslash nLeftrightarrow	&	$\nLeftrightarrow$ \\ \hline
\textbackslash curvearrowright	&	$\curvearrowright$	&	\textbackslash rightrightarrows	&	$\rightrightarrows$	&	\textbackslash Longrightarrow		&	$\Longrightarrow$		&	\textbackslash nleftarrow		&	$\nleftarrow$	\\ \hline
\textbackslash leftharpoondown	&	$\leftharpoondown$	&	\textbackslash leftrightarrows 	& 	$\leftrightarrows$	&	\textbackslash leftrightarrow		&   $\leftrightarrow$		&	\textbackslash nrightarrow		&	$\nrightarrow$	\\ \hline
\textbackslash leftharpoonup	&	$\leftharpoonup$	&	\textbackslash rightleftarrows	&	$\rightleftarrows$ 	&	\textbackslash Leftrightarrow		& 	$\Leftrightarrow$		&	\textbackslash nleftrightarrow	&	$\nleftrightarrow$\\ \hline
\textbackslash rightharpoondown	&	$\rightharpoondown$	&	\textbackslash leftrightharpoons&	$\leftrightharpoons$&	\textbackslash longleftrightarrow	&	$\longleftrightarrow$	&	\textbackslash nearrow			&	$\nearrow$		\\ \hline
\textbackslash rightharpoonup	&	$\rightharpoonup$	&	\textbackslash rightleftharpoons&	$\rightleftharpoons$&	\textbackslash Longleftrightarrow	&	$\Longleftrightarrow$	&	\textbackslash nwarrow			&	$\nwarrow$		\\ \hline
\textbackslash downharpoonleft	&	$\downharpoonleft$	&	\textbackslash leftarrow \rl{یا} \textbackslash gets	&	$\leftarrow$			& 	\textbackslash Lleftarrow	 		&	$\Lleftarrow$			&	\textbackslash searrow			&	$\searrow$		\\ \hline
\textbackslash downharpoonright	&	$\downharpoonright$	&	\textbackslash longleftarrow	&	$\longleftarrow$	& 	\textbackslash Rrightarrow	 		&	$\Rrightarrow$		&	\textbackslash swarrow			&	$\swarrow$		\\ \hline
\textbackslash upharpoonleft	&	$\upharpoonleft$	&	\textbackslash rightarrow \rl{یا} \textbackslash to		&	$\rightarrow$			& 	\textbackslash mapsto 				&	$\mapsto$				&	\textbackslash twoheadleftarrow	&	$\twoheadleftarrow$ \\ \hline
\textbackslash upharpoonright	&	$\upharpoonright$	&	\textbackslash longrightarrow	&	$\longrightarrow$	& 	\textbackslash longmapsto			&	$\longmapsto$			&	\textbackslash twoheadrightarrow&	$\twoheadrightarrow$ \\ \hline
\end{tabular}
\end{latin}
\caption{عملگرهای پیکانی به همراه دستور مربوط به عملگرها}
\label{arrow-operators-table}
\end{table}

\subsubsection{مجموعه‌ای}
عملگرهای مجموعه‌ای مربوط به مجموعه‌هاست که شامل «اجتماع»، «اشتراک»، «عضویت» و
غیره می‌شود. در جدول \ref{set-operators-table} عملگرهای مهم این دسته آورده شده
است.

\begin{table}
\begin{latin}
\centering
\begin{tabular}{|l|c||l|c||l|c||l|c|}
\hline
\rl{دستور}					&	\rl{نمایش}	&	\rl{دستور}					&	\rl{نمایش}		&	\rl{دستور}						&	\rl{نمایش}		&	\rl{دستور}					&	\rl{نمایش}	\\ \hline\hline
\textbackslash cap			&	$\cap$		&	\textbackslash supset 		&	$\supset$		&	\textbackslash varsupsetneqq	&	$\varsupsetneqq$&	\textbackslash Vdash		&	$\Vdash$	\\ \hline
\textbackslash cup			&  	$\cup$		&	\textbackslash supseteq  	& 	$\supseteq$		& 	\textbackslash sqsubset			&	$\sqsubset$		&	\textbackslash Vvdash		&	$\Vvdash$	\\ \hline
\textbackslash sqcap		&	$\sqcap$	&	\textbackslash supseteqq  	&	$\supseteqq$	&	\textbackslash sqsubseteq		&	$\sqsubseteq$	&	\textbackslash nvdash		&	$\nvdash$	\\ \hline
\textbackslash sqcup		&	$\sqcup$	&	\textbackslash supsetneq	&	$\supsetneq$	&	\textbackslash sqsupset			&	$\sqsupset$		&	\textbackslash nvDash		&	$\nvDash$	\\ \hline
\textbackslash bigcap		&	$\bigcap$	&	\textbackslash supsetneqq 	& 	$\supsetneqq$	&	\textbackslash sqsupseteq		&   $\sqsupseteq$	&	\textbackslash nVdash		&	$\nVdash$	\\ \hline
\textbackslash bigcup		&	$\bigcup$	&	\textbackslash nsubseteq	&	$\nsubseteq$ 	&	\textbackslash in				& 	$\in$			&	\textbackslash nVDash		&	$\nVDash$	\\ \hline
\textbackslash bigsqcup		&	$\bigsqcup$	&	\textbackslash nsubseteqq	&	$\nsubseteqq$	&	\textbackslash ni	\rl{یا} \textbackslash owns		&	$\ni$				&	\textbackslash perp			&	$\perp$		\\ \hline
\textbackslash subset		&	$\subset$	&	\textbackslash nsupseteq	&	$\nsupseteq$	&	\textbackslash notin			&	$\notin$		&	\textbackslash mid			&	$\mid$		\\ \hline
\textbackslash subseteq		&	$\subseteq$	&	\textbackslash nsupseteqq	&	$\nsupseteqq$	& 	\textbackslash dashv	 		&	$\dashv$		&	\textbackslash nmid			&	$\nmid$		\\ \hline
\textbackslash subseteqq	&	$\subseteqq$&	\textbackslash varsubsetneq	&	$\varsubsetneq$	& 	\textbackslash vdash	 		&	$\vdash$		&	\textbackslash nparallel	&	$\nparallel$\\ \hline
\textbackslash subsetneq	&	$\subsetneq$&	\textbackslash varsubsetneqq&	$\varsubsetneqq$& 	\textbackslash models 			&	$\models$		&	\textbackslash parallel		&	$\parallel$	\\ \hline
\textbackslash subsetneqq	&	$\subsetneqq$&	\textbackslash varsupsetneq	&	$\varsupsetneq$	& 	\textbackslash vDash 			&	$\vDash$		&								&				\\ \hline
\end{tabular}
\end{latin}
\caption{عملگرهای مجموعه‌ای به همراه دستور مربوط به عملگرها}
\label{set-operators-table}
\end{table}

\subsubsection{متفرقه}
در جدول \ref{other-symbols-table} تعدادی از نمادهایی که در روابط ریاضی و فرمول‌ها استفاده
می‌شوند آورده شده است.

\begin{table}
\begin{latin}
\centering
\begin{tabular}{|l|c||l|c||l|c||l|c|}
\hline
\rl{دستور}					&	\rl{نمایش}	&	\rl{دستور}						&	\rl{نمایش}		&	\rl{دستور}				&	\rl{نمایش}	&	\rl{دستور}					&	\rl{نمایش}	\\ \hline\hline
\textbackslash hbar			&	$\hbar$		&	\textbackslash eth 				&	$\eth$			&	\textbackslash infty	&	$\infty$	&	\textbackslash emptyset		&	$\emptyset$	\\ \hline
\textbackslash hslash		&  	$\hslash$	&	\textbackslash angle  			& 	$\angle$		& 	\textbackslash forall	&	$\forall$	&	\textbackslash varnothing	&	$\varnothing$\\ \hline
\textbackslash ell			&	$\ell$		&	\textbackslash measuredangle  	&	$\measuredangle$&	\textbackslash exists	&	$\exists$	&	\textbackslash triangle		&	$\triangle$	\\ \hline
\textbackslash partial		&	$\partial$	&	\textbackslash \#				&	\#				&	\textbackslash nexists	&	$\nexists$	&	\textbackslash triangledown	&	$\triangledown$	\\ \hline
\textbackslash mho			&	$\mho$		&	\textbackslash \& 				& 	\&				&	\textbackslash therefore&   $\therefore$&	\textbackslash nabla		&	$\nabla$	\\ \hline
\end{tabular}
\end{latin}
\caption{نمادهای متفرقه پرکاربرد در روابط ریاضی به همراه دستور مربوط به آن‌ها}
\label{other-symbols-table}
\end{table}

\subsection{نشانه‌گذاری روی حروف}
گاهی در فرمول‌ها نیاز داریم روی برخی حروف علائمی خاص قرار دهیم. علائمی مثل نقطه،
کلاه\LTRfootnote{Hat}، مد (تیلدا)، پریم و غیره. در جدول \ref{accent-table}
نمونه‌هایی از این نوع علامت‌گذاری‌ها روی حروف و کلمات مشاهده می‌شود.

\begin{table}
\begin{latin}
\centering
\begin{tabular}{|l|c||l|c||l|c||l|c|}
\hline
\rl{دستور}					&	\rl{نمایش}	&	\rl{دستور}					&	\rl{نمایش}	&	\rl{دستور}						&	\rl{نمایش}			&	\rl{دستور}									&	\rl{نمایش}					\\ \hline\hline
\textbackslash acute\{x\}	&	$\acute{x}$	&	\textbackslash vec\{x\}		&	$\vec{x}$	&	\textbackslash widetilde\{xxx\} &	$\widetilde{xxx}$	&	\textbackslash overleftarrow\{xxx\}			&	$\overleftarrow{xxx}$		\\ \hline
\textbackslash grave\{x\}	&  	$\grave{x}$	&	\textbackslash dot\{x\}		& 	$\dot{x}$	& 	\textbackslash widehat\{xxx\}	&	$\widehat{xxx}$		&	\textbackslash underleftarrow\{xxx\}		&	$\underleftarrow{xxx}$		\\ \hline
\textbackslash tilde\{x\}	&	$\tilde{x}$	&	\textbackslash ddot\{x\} 	&	$\ddot{x}$	&	\textbackslash overbrace\{xxx\} &	$\overbrace{xxx}$	&	\textbackslash overrightarrow\{xxx\}		&	$\overrightarrow{xxx}$		\\ \hline
\textbackslash bar\{x\}		&	$\bar{x}$	&	\textbackslash dddot\{x\}	&	$\dddot{x}$	&	\textbackslash underbrace\{xxx\}&	$\underbrace{xxx}$	&	\textbackslash underrightarrow\{xxx\}		&	$\underrightarrow{xxx}$		\\ \hline
\textbackslash breve\{x\}	&  	$\check{x}$	&						  		& 				& 	\textbackslash overline\{xxx\}	&	$\overline{xxx}$	&	\textbackslash overleftrightarrow\{xxx\}	&	$\overleftrightarrow{xxx}$	\\ \hline
\textbackslash hat\{x\}		&  	$\hat{x}$	&								& 				& 	\textbackslash underline\{xxx\}	&	$\underline{xxx}$	&	\textbackslash underleftrightarrow\{xxx\}	&	$\underleftrightarrow{xxx}$	\\ \hline
\end{tabular}
\end{latin}
\caption{نشانه‌گذاری‌های مختلف روی حروف، به همراه دستورات مربوطه}
\label{accent-table}
\end{table}

اگرچه در این مثال‌ها از حرف \lr{x} استفاده شده است اما به جای آن می‌توان هر حرف
دیگری را قرار داد (به عنوان مثال دستور \lr{\textbackslash tilde\{O\}} این
نتیجه را خواهد داشت: $\tilde{O}$).

\subsection{توابع}
منظور از توابع، توابع شناخته شده ریاضی هستند که معمولا به صورت یک کلمه در روابط
نوشته می‌شوند. به عنوان مثال توابع مثلثاتی: سینوس (\lr{sin})، کسینوس (\lr{cos})
تانژانت (\lr{tan}) و \lr{\dots} و توابع لگاریتمی (\lr{log} یا \lr{ln}) و نمایی
(\lr{exp}) از این موارد هستند. بسیاری از این توابع به صورت دستور از پیش تعریف
شده در \lr{\LaTeX} وجود دارند. نمونه‌ای از یک رابطه ریاضی که از این توابع
استفاده می‌کند در زیر مشاهده می‌شود:
\begin{latex}
\[
 \exp(i \phi) = \cos(\phi) + i \sin(\phi)
\]
\end{latex}
\[
 \exp(i \phi) = \cos(\phi) + i \sin(\phi)
\]

در جدول \ref{function-table} نمونه‌هایی از دستورات تعریف شده برای توابع ریاضی آورده شده
است.

\begin{table}
\begin{latin}
\centering
\begin{tabular}{|l|c||l|c||l|c||l|c|}
\hline
\rl{دستور}				&	\rl{نمایش}	&	\rl{دستور}			&	\rl{نمایش}	&	\rl{دستور}			&\rl{نمایش}	&	\rl{دستور}				&	\rl{نمایش}	\\ \hline\hline
\textbackslash sin		&	$\sin$		&	\textbackslash sinh	&	$\sinh$		&	\textbackslash arg 	&	$\arg$	&	\textbackslash lg		&	$\lg$		\\ \hline
\textbackslash cos		&  	$\cos$		&	\textbackslash cosh	& 	$\cosh$		& 	\textbackslash deg	&	$\deg$	&	\textbackslash ln		&	$\ln$		\\ \hline
\textbackslash tan		&	$\tan$		&	\textbackslash tanh &	$\tanh$		&	\textbackslash det	&	$\det$	&	\textbackslash log		&	$\log$		\\ \hline
\textbackslash cot		&	$\cot$		&	\textbackslash coth	&	$\coth$		&	\textbackslash dim	&	$\dim$	&	\textbackslash lim		&	$\lim$		\\ \hline
\textbackslash arcsin	&  	$\arcsin$	&	\textbackslash sec	& 	$\sec$		& 	\textbackslash gcd	&	$\gcd$	&	\textbackslash liminf	&	$\liminf$	\\ \hline
\textbackslash arccos	&  	$\arccos$	&	\textbackslash csc	& 	$\csc$		& 	\textbackslash inf	&	$\inf$	&	\textbackslash limsup	&	$\limsup$	\\ \hline
\textbackslash arctan	&	$\arctan$	&	\textbackslash max	&	$\max$		&	\textbackslash sup	&	$\sup$	&	\textbackslash varliminf&	$\varliminf$\\ \hline
						&  				&	\textbackslash min	& 	$\min$		& 	\textbackslash exp	&	$\exp$	&	\textbackslash varlimsup&	$\varlimsup$\\ \hline
\end{tabular}
\end{latin}
\caption{توابعی که در \lr{\LaTeX} به صورت دستور از پیش تعریف شده وجود دارند.}
\label{function-table}
\end{table}

برای برخی از عملگرها و توابع مثل تابع حد (\lr{lim}) زیرنویس زیر کلمه مربوط به
تابع قرار می‌گیرد. در زیر نمونه‌ای از زیرنویس، برای تابع حد را مشاهده می‌کنید:

\begin{latex}
\[
 \lim_{x \to \infty} \exp(-x) = 0
\]
\end{latex}
\[
 \lim_{x \to \infty} \exp(-x) = 0
\]

\subsection{کسر و دوجمله‌ای}
با استفاده از دستور \lr{\textbackslash frac\{\textit{up}\}\{\textit{down}\}}
می‌توان یک کسر ایجاد کرد. به جای \lr{\textit{up}} عبارت صورت و به جای
\lr{\textit{down}} مخرج کسر قرار داده می‌شود. ضریب دوجمله‌ای (که به آن تابع
انتخاب نیز می‌گویند) را نیز می‌توان با استفاده از دستور \lr{\textbackslash
binom\{\textit{up}\}\{\textit{down}\}} در فرمول وارد کرد. در زیر مثالی از این دو
دستور آورده شده است:
\begin{latex}
\[
 \frac{n!}{k!(n-k)!} = \binom{n}{k}
\]
\end{latex}
\[
 \frac{n!}{k!(n-k)!} = \binom{n}{k}
\]
	
\begin{note}
برای ضریب دوجمله‌ای می‌توان از دستور \lr{\textbackslash choose} هم استفاده کرد.
این دستور نیازی به بسته \lr{amsmath} ندارد. در حالیکه دستور \lr{\\binom} در بسته
\lr{\\amsmath} تعریف شده است. برای کسرها نیز می‌توان از دستور \lr{\\over}
استفاده کرد که نیازی به بسته \lr{amsmath} ندارد. در حالیکه دستور \lr{\\frac} در
بسته \lr{amsmath} تعریف شده است.
دستورات \lr{\\choose} و \lr{\\over} به صورت زیر
استفاده می‌شوند:
\end{note}
\begin{latex}
\[
  {n! \over k!(n-k)!} = {n \choose k}
\]
\end{latex}
\[
  {n! \over k!(n-k)!} = {n \choose k}
\]
دستور \lr{\textbackslash frac} را می‌توان به صورت تودرتو نیز به کار برد. به
عنوان مثال برای مواقعی که صورت یک کسر خودش یک کسر دیگر است. به مثال زیر توجه
کنید:
\begin{latex}
\[
 \frac{\frac{1}{x}+\frac{1}{y}}{y-z}
\]
\end{latex}
\[
 \frac{\frac{1}{x}+\frac{1}{y}}{y-z}
\]

\begin{note}
برای کسرهای نسبتا ساده می‌توان کسر را به صورتی زیباتری نشان داد. برای این کار از
دستورات توان و اندیس استفاده می‌شود. مثال زیر نمونه‌ای از این نوع کسر است:
\end{note}
\begin{latex}
\[
 ^3/_7
\]
\end{latex}
\[
 ^3/_7
\]

\subsection{جذر و ریشه}
برای نوشتن جذر در فرمول‌ها از دستور \lr{\textbackslash sqrt} استفاده می‌شود. این
دستور روی عبارت قرار گرفته بین آکولادها علامت رادیکال قرار می‌دهد.
دستور \lr{\textbackslash sqrt} یک آرگومان اختیاری هم دارد که با استفاده از آن
می‌توان ریشه رادیکال را تعیین کرد. به مثال‌های زیر توجه کنید:
\begin{latex}
\[
\sqrt{\frac{a}{b}}
\]
\end{latex}
\[
\sqrt{\frac{a}{b}}
\]
\begin{latex}
\[
\sqrt[n]{1+x+x^2+x^3+\ldots}
\]
\end{latex}
\[
\sqrt[n]{1+x+x^2+x^3+\ldots}
\]
	
\subsection{جمع و انتگرال}
برای وارد کردن علامت جمع (سیگما) از دستور \lr{\textbackslash sum} استفاده
می‌شود. برای تعیین کران‌های بالا و پایین این دستور نیز از عملگرهای اندیس و
توان می‌توان استفاده کرد. عملگر اندیس (\lr{$\_$}) برای تعیین کران پایین و عملگر
توان ( \lr{$\^$} ) برای تعیین کران بالا. به مثال زیر توجه کنید:

\begin{latex}
\[
 \sum_{i=1}^{10} t_i
\]
\end{latex}
\[
 \sum_{i=1}^{10} t_i
\]

برای وارد کردن انتگرال نیز از دستور \lr{\textbackslash int} استفاده می‌شود. برای
تعیین کران بالا و پایین این دستور نیز می‌توان از عملگرهای اندیس و توان استفاده
کرد. علاوه بر این، برای نمایش متغیر انتگرال‌گیری همراه با علامت \lr{d} بهتر
است از دستور \lr{\textbackslash mathrm\{\}} استفاده شود. در زیر نمونه‌ای از
دستور انتگرال آورده شده است:
\begin{latex}
\[
 \int_0^\infty e^{-x} \mathrm{d}x
\]
\end{latex}
\[
 \int_0^\infty e^{-x} \mathrm{d}x
\]
در جدول \ref{sum-and-integral-table} نمونه‌های دیگری از این عملگرها به همراه
دستوراتشان آورده شده است.

\begin{table}
\begin{latin}
\centering
\begin{tabular}{|l|c||l|c||l|c||l|c|}
\hline
\rl{دستور}				&	\rl{نمایش}	&	\rl{دستور}				&	\rl{نمایش}		&	\rl{دستور}				&\rl{نمایش}		&	\rl{دستور}				&	\rl{نمایش}	\\ \hline\hline
\textbackslash sum		&	$\sum$		&	\textbackslash bigotimes&	$\bigotimes$	&	\textbackslash bigsqcup &	$\bigsqcup$	&	\textbackslash iint		&	$\iint$		\\ \hline
\textbackslash prod		&  	$\prod$		&	\textbackslash bigodot	& 	$\bigodot$		& 	\textbackslash bigsqcup	&	$\bigsqcup$	&	\textbackslash iiint	&	$\iiint$	\\ \hline
\textbackslash coprod	&	$\coprod$	&	\textbackslash bigcup 	&	$\bigcup$		&	\textbackslash int		&	$\int$		&	\textbackslash iiiint	&	$\iiiint$	\\ \hline
\textbackslash bigoplus	&	$\bigoplus$	&	\textbackslash bigcap	&	$\bigcap$		&	\textbackslash oint		&	$\oint$		&	\textbackslash idotsint	&	$\idotsint$	\\ \hline
\end{tabular}
\end{latin}
\caption{انواع عملگرهای جمع و انتگرال و عملگرهای مشابه آن‌ها به همراه دستور
مربوط به هر یک.}
\label{sum-and-integral-table}
\end{table}

\subsection{کمانک‌ها و پرانتزها}
در بسیاری از فرمول‌ها استفاده از پرانتز، براکت و سایر کمانک‌ها ضروری است. بدون
استفاده از پرانتز بسیاری از فرمول‌ها مبهم خواهند بود. به علاوه برخی ساختارهای
ریاضی مثل ماتریس‌ها بین چنین جداکننده‌هایی قرار می‌گیرند. انواع مختلفی از
این علائم وجود دارند که برخی از آن‌ها در جدول \ref{parantesis-brakets-table}
آورده شده‌اند.

\begin{table}
\begin{latin}
\centering
\begin{tabular}{|l|c||l|c||l|c|}
\hline
\rl{دستور}														&	\rl{نمایش}	&	\rl{دستور}													&	\rl{نمایش}				&	\rl{دستور}												&	\rl{نمایش}					\\ \hline\hline
( a )															&	$( a )$		&	\textbackslash| e \textbackslash| \rl{یا} \textbackslash lVert e \textbackslash rVert &	$\| e \|$				&	\textbackslash ulcorner i \textbackslash urcorner		&	$\ulcorner i \urcorner$		\\ \hline
[ b ]															&  	$[ b ]$		&	\textbackslash langle f \textbackslash rangle				& 	$\langle f \rangle$		& 	\textbackslash lgroup j \textbackslash rgroup			&	$\lgroup j \rgroup$			\\ \hline
\{ c \}	\rl{یا} \textbackslash lbrace c \textbackslash rbrace	&	$\{ c \}$	&	\textbackslash lfloor g \textbackslash rfloor				&	$\lfloor g \rfloor$		&	\textbackslash lmoustache k \textbackslash rmoustache	&	$\lmoustache k \rmoustache$	\\ \hline
| d |	\rl{یا} \textbackslash lvert d \textbackslash rvert		&	$| d |$		&	\textbackslash lceil h \textbackslash rceil 				&	$\lceil h \rceil$		&															&								\\ \hline
\end{tabular}
\end{latin}
\caption{انواع مختلفی از پرانتزها، براکت‌ها و کمانک‌ها به همراه دستور مربوط به هر یک.}
\label{parantesis-brakets-table}
\end{table}

% Automatic sizing
\subsubsection{اندازه کمانک‌ها و پرانتزها}
\label{paranteses-and-brakets-subsubsection}
در اغلب موارد نیاز داریم بزرگی یا کوچکی کمانک یا پرانتز مورد استفاده با توجه
به فرمول تعیین شود. به عنوان مثال فرض کنید یک کسر داریم و می‌خواهیم هنگامی که
دور این کسر پرانتز قرار می‌دهیم اندازه پرانتز به گونه‌ای باشد که با کسر تناسب
داشته باشد. در \lr{\LaTeX} با استفاده از دستورات \lr{\textbackslash left} و
\lr{\textbackslash right} این کار به صورت خودکار انجام می‌شود. به مثال‌های زیر
توجه کنید. دقت شود که چگونه با استفاده از این دو دستور اندازه کمانک به کار رفته
متناسب با فرمول تغییر می‌یابد.
\begin{latex}
\[
 \left( \frac{x^2}{y^3} \right) , ( \frac{x^2}{y^3} ) ,
 \left\{ \frac{x^2}{y^3} \right\} , \{ \frac{x^2}{y^3} \} ,
 \left \lmoustache \frac{k}{\frac{l}{m}} \right \rmoustache ,
 \left \lgroup \frac{k}{\frac{l}{m}} \right \rgroup 
\]
\end{latex}

\[
 \left( \frac{x^2}{y^3} \right) , ( \frac{x^2}{y^3} ) ,
 \left\{ \frac{x^2}{y^3} \right\} , \{ \frac{x^2}{y^3} \} ,
 \left \lmoustache \frac{k}{\frac{l}{m}} \right \rmoustache ,
 \left \lgroup \frac{k}{\frac{l}{m}} \right \rgroup  
\]

تقریبا تمام کمانک‌های جدول \ref{parantesis-brakets-table} را می‌توان همراه با
دستورات \lr{\textbackslash left} و \lr{\textbackslash right} به کار برد.

در صورتی که بخواهیم فقط در یک طرف از یک فرمول یک کمانک قرار دهیم و در طرف دیگر
آن کمانک متناظر آن را قرار ندهیم می‌توان از یک علامت نقطه به جای کمانکی که باید
حذف شود استفاده کرد. مثال زیر نمونه‌ای از این موضوع را نشان می‌دهد:
\begin{latex}
\[
 \left.\frac{x^3}{3}\right|_0^1
\]
\end{latex}

\[
 \left. \frac{x^3}{3} \right|_0^1
\]

حتی ممکن است کمانک به کار رفته در یک طرف با کمانک به کار رفته در طرف دیگر همسان
نباشد:

\begin{latex}
\[
 \left| \frac{x + y}{\sqrt{2}} \right \rangle
\]
\end{latex}

\[
 \left| \frac{x + y}{\sqrt{2}} \right \rangle
\]

% Manual sizing
علاوه بر تغییر اندازه خودکار که با دستورات \lr{\textbackslash left} و
\lr{\textbackslash right} انجام می‌شود امکان تغییر اندازه کمانک‌ها به صورت دستی
نیز وجود دارد. دستورات \lr{\textbackslash big} ، \lr{\textbackslash Big} ،
\lr{\textbackslash bigg} و \lr{\textbackslash Bigg} بدین منظور استفاده می‌شوند.
در زیر مثال‌هایی از استفاده از این دستورات آورده شده است:
\begin{latex}
\[
 ( \big( \Big( \bigg( \Bigg(
\]
\end{latex}
\[
 ( \big( \Big( \bigg( \Bigg(
\]
\begin{latex}
\[
 \lfloor \big\lfloor \Big\lfloor \bigg\lfloor \Bigg\lfloor
\]
\end{latex}
\[
 \lfloor \big\lfloor \Big\lfloor \bigg\lfloor \Bigg\lfloor
\]

% Matrices and arrays
\subsection{ماتریس و آرایه}
در کلی‌ترین حالت برای نوشتن یک ماتریس از محیط \lr{matrix} در \lr{\LaTeX} استفاده
می‌شود. برای جداکردن سطرها ازعلامت (\lr{\textbackslash\textbackslash}) و برای
قرار دادن درایه‌های یک سطر در ستون‌های مختلف از علامت $\&$ استفاده می شود.
\begin{latex}
\[
 \begin{matrix}
  a & b & c \\
  d & e & f \\
  g & h & i
 \end{matrix}
\]
\end{latex}
\[
 \begin{matrix}
  a & b & c \\
  d & e & f \\
  g & h & i
 \end{matrix}
\]

% برای هم راستا کردن درایه‌های ماتریس می‌توان از محیط \lr{matrix$^*$} استفاده کرد.
% در زیر مثالی از این دو محیط آورده شده است. دقت شود که داده‌های ماتریس در محیط
% \lr{matrix$^*$} چگونه تراز شده‌اند.
% \begin{latex}
% \[
%  \begin{matrix}
%   -1 & 3 \\
%   2 & -4
%  \end{matrix}
%  =
%  \begin{matrix*}[r]
%   -1 & 3 \\
%   2 & -4
%  \end{matrix*}
% \]
% \end{latex}
% \[
%  \begin{matrix}
%   -1 & 3 \\
%   2 & -4
%  \end{matrix}
%  =
%  \begin{matrix*}[r]
%   -1 & 3 \\
%   2 & -4
%  \end{matrix*}
% \]

معمولا ماتریس‌ها بین دو پرانتز یا نوع دیگری از کمانک‌ها قرار داده می‌شوند. برای
این کار طبق آنچه در قسمت \ref{paranteses-and-brakets-subsubsection} گفته شد از
دستورات \lr{\textbackslash left} و \lr{\textbackslash right} استفاده می‌شود. به
عنوان مثال در کد زیر ماتریس قبل را بین دو پرانتز و دو کروشه قرار داده‌ایم:

\begin{latex}
\[
 \left(
 \begin{matrix}
  a & b & c \\
  d & e & f \\
  g & h & i
 \end{matrix}
 \right)
 \left[
 \begin{matrix}
  a & b & c \\
  d & e & f \\
  g & h & i
 \end{matrix}
 \right]
\]
\end{latex}
\[
 \left(
 \begin{matrix}
  a & b & c \\
  d & e & f \\
  g & h & i
 \end{matrix}
 \right)
 \left[
 \begin{matrix}
  a & b & c \\
  d & e & f \\
  g & h & i
 \end{matrix}
 \right]
\]

علاوه بر روش بالا برای قرار دادن پرانتز یا سایر کمانک‌ها در اطراف یک ماتریس،
محیط‌های از پیش‌تعریف شده‌ای نیز در \lr{\LaTeX} وجود دارند که هر یک به صورت
خودکار علامت خاصی را اطراف ماتریس قرار می‌دهد. این محیط‌ها و علامت‌هایی که
استفاده می‌کنند در جدول \ref{matrix-environments} آورده شده است.

\begin{table}
\begin{latin}
\centering
\begin{tabular}{|l|c||l|c||l|c|}
\hline
\rl{محیط}	&	\rl{کمانک}	&	\rl{محیط}	&	\rl{کمانک}	&	\rl{محیط}	&	\rl{کمانک} \\ \hline\hline 
pmatrix		&	$(\,)$		&	Bmatrix		&	$\{\,\}$	&	Vmatrix 	&	$\|\,\|$	\\ \hline
pmatrix*	&  	$(\,)$		&	Bmatrix*	& 	$\{\,\}$	& 	Vmatrix*	&	$\|\,\|$	\\ \hline
bmatrix		&	$[\,]$		&	vmatrix 	&	$|\,|$		&				&				\\ \hline
bmatrix*	&	$[\,]$		&	vmatrix*	&	$|\,|$		&				&				\\ \hline
\end{tabular}
\end{latin}
\caption{محیط‌های مختلف از پیش‌تعریف شده در \lr{\LaTeX} برای ماتریس‌ها.}
\label{matrix-environments}
\end{table}

\begin{note}
تفاوت محیط‌های بدون ستاره و ستاره‌دار این است که در محیط‌های بدون ستاره ستون‌های
ماتریس به صورت پیش‌فرض وسط ستون قرار می‌گیرند در حالی که در محیط‌های بدون
ستاره‌دار امکان تغییر تراز عناصر ستونی ماتریس وجود دارد.
\end{note}
بسیاری از مواقع ماتریس‌هایی در فرمول‌ها داریم که به جای نوشتن تمام عناصر آن (در
سطر یا ستون) سه نقطه قرار می‌دهیم. برای نوشتن چنین ماتریس‌هایی می‌توان از
دستورات \lr{\textbackslash cdot}، \lr{\textbackslash vdot} و \lr{\textbackslash
ddot} استفاده کرد:
\begin{latex}
\[
 A_{m,n} =
 \begin{pmatrix}
  a_{1,1} & a_{1,2} & \cdots & a_{1,n} \\
  a_{2,1} & a_{2,2} & \cdots & a_{2,n} \\
  \vdots  & \vdots  & \ddots & \vdots  \\
  a_{m,1} & a_{m,2} & \cdots & a_{m,n}
 \end{pmatrix}
\]
\end{latex}
\[
 A_{m,n} =
 \begin{pmatrix}
  a_{1,1} & a_{1,2} & \cdots & a_{1,n} \\
  a_{2,1} & a_{2,2} & \cdots & a_{2,n} \\
  \vdots  & \vdots  & \ddots & \vdots  \\
  a_{m,1} & a_{m,2} & \cdots & a_{m,n}
 \end{pmatrix}
\]
%Matrices in running text
\subsubsection{درج ماتریس در متن}
برای درج یک ماتریس کوچک بین کلمات یک متن محیطی تحت عنوان \lr{smallmatrix}
وجود دارد. مثال زیر یک ماتریس را درون یک متن و بین کلمات آن قرار
می‌دهد:
\begin{latex}
This is a small matrix: 
$\bigl(\begin{smallmatrix}
a&b\\ c&d
\end{smallmatrix} \bigr)$.
This matrix is a small matrix suitable for use in text.
\end{latex}
\begin{latin}
This is a small matrix: 
$\bigl(\begin{smallmatrix}
a&b\\ c&d
\end{smallmatrix} \bigr)$.
This matrix is a small matrix suitable for use in text.
\end{latin}


% %     10.1 Matrices in running text
% \subsection{\lr{Adding text to equations}}
% %     11.1 Formatted text
% \subsection{\lr{Formatting mathematics symbols}}
%     12.1 Accents
%\subsection{\lr{Controlling horizontal spacing}}
%\subsection{\lr{Advanced Mathematics: AMS Math package}}
%     15.1 Introducing text and dots in formulas
%     15.2 Dots
%     15.3 Write an equation with the align environment
% برای کسب اطلاعات بیشتر در مورد نماد های استفاده شده در \lr{LATEX} می توانید به آدرس زیر مراجعه کنید:
% \href{http://www.wikibooks.org}{Mathematics}
\subsection{سایر محیط‌های فرمول‌نویسی}
\label{sec:formula-environments}
در قسمت قبل نحوه فرمول‌نویسی به سبک \lr{\LaTeX} شرح داده شد و دو محیط اصلی برای
فرمول‌نویسی معرفی گردید که عبارتند از: \lr{displaymath} و \lr{equation*} و البته
برای نوشتن فرمول‌ها در میان خطوط متنی نیز از دو علامت \lr{\textbackslash $\$$}
برای ابتدا و انتها استفاده می‌شد. در \lr{\LaTeX} برای فرمول‌نویسی محیط‌های دیگری
نیز وجود دارند. در این قسمت این محیط‌ها و ویژگی‌های آن‌ها معرفی شده است.

یکی از مهم‌ترین محدودیت‌های محیط‌های \lr{displaymath} و \lr{equation*} این است
که نمی‌توان فرمول‌های چند خطی را با استفاده از آن‌ها نمایش داد. برای نمایش یک
فرمول چند خطی می‌توان از محیط‌های \lr{align} و \lr{align*} استفاده کرد. در این
محیط‌ها برای رفتن به یک خط جدید از دو علامت بک‌اسلش (\lr{\textbackslash\textbackslash})
استفاده می‌شود. در زیر نمونه‌ای از یک فرمول چند خطی در محیط \lr{align*} نمایش
داده شده است:
\begin{latex}
\begin{align*}
 f(x) &= (x+a)(x+b) \\
 &= x^2 + (a+b)x + ab
\end{align*}
\end{latex}
\begin{align*}
 f(x) &= (x+a)(x+b) \\
 &= x^2 + (a+b)x + ab
\end{align*}

\begin{info}
علامت $\&$ در این محیط خطوط مختلف فرمول را تراز می‌کند. در واقع قوانین این محیط
خیلی شبیه به قوانین نوشتن جدول و ماتریس است. در جدول‌ها و ماتریس‌ها نیز برای
رفتن به سطر جدید از علامت \lr{\textbackslash\textbackslash} و برای تراز کردن
اطلاعات از علامت $\&$ استفاده می‌شود.
\end{info}

محیط‌های دیگری نیز برای فرمول‌نویسی در \lr{\LaTeX} وجود دارد که هر یک ویژگی‌ها
و امکانات خاص خود را دارند. در جدول \ref{formulla-environments-table} این
محیط‌ها و توصیفی خلاصه از هر یک آورده شده است.

\begin{table}
% \begin{latin}
\centering
\begin{tabular}{|r|p{8cm}|}
\hline
محیط							&	توصیف		\\ \hline\hline
\lr{eqnarray} و \lr{eqnarray*}	&	این محیط مشابه محیط \lr{align} و \lr{aling*}
عمل می‌کند. البته استفاده از این محیط توصیه نمی‌شود زیرا در فاصله‌گذاری باعث
ناسازگاری‌هایی می‌شود \\ \hline 
\lr{multline} و \lr{multline*}	&  	در این محیط خط اول از سمت چپ و خط آخر از سمت
راست تراز می‌شود.		\\ \hline
\lr{gather} و \lr{gather*}		&	در این محیط فرمول‌ها به صورت متوالی و بدون
ترازبندی قرار داده می‌شوند.		\\ \hline
\lr{flalign} و \lr{flalign*}	&	این محیط مشابه محیط	\lr{align} است با این تفاوت
که در هر خط، اولین فرمول از سمت چپ و آخرین فرمول از سمت راست ترازبندی می‌شود	\\\hline
\lr{alignat} و \lr{alignat*}	&	این محیط یک آرگومان به عنوان ورودی می‌گیرد که
تعیین کننده تعداد ستون‌هاست. و همچنین در این محیط می‌توان به صورت صریح فاصله
افقی بین فرمول‌ها را تعیین کرد. تعداد ستون‌ها را می‌توان با شمارش علامت‌های $\&$
به اضافه یک تقسیم بر ۲ به دست آورد.	\\ \hline
\end{tabular} 
% \end{latin}
\caption{برخی از محیط‌های فرمول‌نویسی تعریف شده در \lr{\LaTeX}.}
\label{formulla-environments-table}
\end{table} 


% NOTE: هادی ۵-۱۳۹۱: در این قسمت در مورد فرمول‌نویسی در latex صحبت می‌شود.

\section{محیط‌های خاص}


note

warning

bug

todo

attention

brief

detail

verbatin

test

code


% NOTE: هادی ۵-۱۳۹۱: در این قسمت در مورد فرمول‌نویسی مناسب برای doxygen صحبت می‌شود.