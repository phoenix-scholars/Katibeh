% \glspl{doxygen:tag}‌های در این قسمت تشریح می‌شود که برای نوشتن متن مستقل از خود جای آن و منظور
% آن نوشته می شود

\section{متن مستند}

دسته‌ای از \glspl{doxygen:tag}‌ها برای ایجاد متن‌های متفاوت مورد استفاده قرار می‌گیرد. هدف
اصلی استفاده از این \glspl{doxygen:tag}‌ها زیبا کردن متن مستند است و کاربرد دیگری ندارد. برای
نمونه نوشتن یک متن با قلم‌های زخیم‌تر نمونه‌ای از این کاربردها است. علاوه بر این
نیز دسته‌ای از برچبس‌ها برای الحاق تصویر مورد استفاده قرار می‌گیرد. در این بخش به بررسی
پرکاربردترین \glspl{doxygen:tag}‌های پرداخته شده است که در این نوع کاربردها مورد استفاده قرار
می‌گیرد.

% Text em p c a 

برای نمایش یک کلمه با قلم‌های زخیم‌تر (که معمولا برای نمایش واژه‌های کلید مورد
استفاده قرار می‌گیرد) از \glspl{doxygen:tag} \lr{em} استفاده می‌شود. ساختار کلی
این \glspl{doxygen:tag} به صورت زیر است.

\begin{doxygen}
\em <word>
\end{doxygen}

برای نمایش چندین واژه به عنوان واژه‌های مهم نمی‌توان از این \glspl{doxygen:tag}
یا معادل با آن استفاده کرد. کاربرد این \glspl{doxygen:tag} تنها زمانی که است که
یک واژه مد نظر باشد. برای نمایش چندین واژه با استفاده از قلم‌های زخیم‌تر می‌توان
از ساختار زیر استفاده کرد.

\begin{doxygen}
<tt> Multi Emphasize Words </tt>
\end{doxygen}

در بسیار از موارد نیاز است که در متن مستند به پارامترهای ورودی توابع اشاره کرد.
در این حالت سعی می‌شود نام پارامترها از دیگر واژه‌ها متفاوت باشد.
در این موارد نه تنها می‌توان از \lr{em} بلکه از \glspl{doxygen:tag}هایی مانند
\lr{p} و  \lr{a} نیز استفاده کرد. این \glspl{doxygen:tag}ها به ترتیب تداییگر
واژه‌های \lr{parameter} و \lr{argument} است.

\begin{note}
استفاده از \lr{a}، \lr{p} و \lr{em} همگی مشابه به یک دیگر بوده و لزومی ندارد
که پارامتر ورودی این \glspl{doxygen:tag}‌ها نام یک پارامتر از تابع باشد.
\end{note}

نه تنها \glspl{doxygen:tag} \lr{em} بلکه تمام \glspl{doxygen:tag}‌های معادل آن با استفاده از روش مشابهی
واژه‌های ورودی را متمایز از دیگر واژه‌ها نمایش می‌دهند. یکی دیگر از \glspl{doxygen:tag}‌ها که
تنها برای نمایش زخیم‌تر یک واژه مورد استفاده قرار می‌گیرد \glspl{doxygen:tag} \lr{b} است که
خلاصه شده \lr{bold} است. ساختار کلی این \glspl{doxygen:tag} نیز به صورت زیر است.

\begin{doxygen}
\b <word>
\end{doxygen}

در اینجا نیز برای نمایش چندین واژه به صورت ذخیم‌تر می‌توان از ساختار زیر استفاده
کرد.

\begin{doxygen}
<b> Multi Emphasize Words </b>
\end{doxygen}

% TODO: maso: یک نمونه کلی از \glspl{doxygen:tag}‌های متن آورده شود

% Image

با استفاده از \glspl{doxygen:tag} \rl{image} می‌توان یک شکل را در مستند اضافه
کرد. ساختار کلی این ورچسب به صورت زیر است.
 
\begin{doxygen}
\image <format> <file> ["caption"] [<sizeindication>=<size>]
\end{doxygen}

در این \glspl{doxygen:tag} از یک پارامتر برای تعیین قالب خروجی استفاده
می‌شود از این رو برای استفاده از شکل در قالب‌های متفاوت خروجی می‌بایست این
\glspl{doxygen:tag} را با استفاده از قالب‌های متفاوتی به کار برد. برای نمونه
اگر بخواهید یک تصویر هم در قالب \glspl{html} و هم \glspl{latex} اضافه شود باید
دو بار این ورچسب را استفاده کرد، یکی برای اضافه کردن تصویر در خروجی \glspl{html}
و دیگری برای \glspl{latex}.

همانگونه که گفته شد برای اضافه کردن یک تصویر در هر قالب خروجی باید نوع آن را
برای این دستور تعیین کرد. نخستین پارامتر این دستور تعیین قالب خروجی است که
می‌تواند یکی از مقادیر زیر باشد:

\begin{itemize}
  \item \lr{html}
  \item \lr{latex}
  \item \lr{rtf}
\end{itemize}

پارامتر دوم از این \glspl{doxygen:tag} نام پرونده‌ را تعیین می‌کند.  این پرونده
در مسیری جستجو می‌شود که در تنظیم‌های مستند آورده شده است. این مسیر با استفاده
از \glspl{doxygen:tag} \lr{IMAGE\_PATH} تعیین می‌شود. در فرآیند تولید مستند،
پرونده شکل از مسیرهایی که تعیین شده در مسیرهای تعیین شده برای خروجی‌های متفاوت
\glspl{computer::copy} شده و در مستند‌های ایجاد شده وارد می‌شود. البته زمانی که تصاویر در
خارج از مسیر مستند سازی قرار دارد نیز می‌توان با استفاده از مسیر کامل شکل مورد
نظر را به مستند اضافه کرد. در این روش از \lr{URL} برای مشخص کردن مسیر پروند
تصویر استفاده می‌شود از این رو امکان اضافه کردن تصاویر از مخزن‌های شبکه نیز ممکن می‌باشد.

ساختار کلید تعیین مسیر شکل‌ها در پیکره بندی سیستم نیز به صورت زیر زیر است.

\begin{doxygen}
IMAGE_PATH = my_image_dir
\end{doxygen}

\begin{note}
زمانی که نام پرونده شامل فضای خالی است باید نام آن را به صورت کامل در کوت
قرار دهید.
\end{note}

سومین پارامتر این دستور متنی را تعیین می‌کند که در نمایش تصویر مورد
استفاده قرار می‌گیرد. این پارامتر دلخاه بوده و برای اضافه کردن یک تصویر به مستند
لزومی به استفاده از آن نیست. این پارامتر باید در یک خط نوشته شده و در هر حال در
کوت قرار گیرد حتی اگر شامل فضای خالی نباشد.

\begin{note}
کوت‌های استفاده شده در متن تصویر در خروجی نمایش داده نخواهد شد و صرفا برای
تعیین متن در دستور مورد استفاده قرار می‌گیرد.
\end{note}

اخرین پارامتر این \glspl{doxygen:tag} خصوصیت‌های عرض و ارتفاع تصویر در نمایش را
تعیین می‌کند. این پارامترها نیز کاملا اختیاری است. خصوصیت‌های مجاز در این دستور
عبارت‌اند از:

\begin{itemize}
  \item \lr{width}
  \item \lr{hieght}
\end{itemize}

که به ترتیب عرض و ارتفاع تصویر را تعیین می‌کنند. استفاده از این خصوصیت‌ها در
قالب \glspl{latex} بسیار مهم بوده و منجر به نتایج مناسب در خروجی می‌شود.
اندازه‌های تعیین شده برای این پارامترها باید بر اساس استانداردهایی باشد که در قالب خروجی مورد
استفاده قرار می‌گیرد.

اضافه کردن یک تصویر در قالب خروجی لیتک به گونه‌ای که عرض آن برابر با نصف اندازه
برگه بوده و ارتفاع آن به صورت خودکار تعیین شود به صورت زیر عمل می‌شود:

\begin{Java}
/**
 * Here is an application arch. 
 *  \image latex application.eps "My application" width=0.5\textwidth
 */
\end{Java}

بر اساس این \glspl{document block} مستند \glspl{latex} به صورت زیر تولید خواهد
شد.

\begin{latex}
Here is an application arch.

\begin{figure}
\includegraphics[width=0.5\textwidth]{application.eps}
\caption{My application}
\end{figure}
\end{latex}

اما در خروجی‌ها با قالب‌های دیگر مانند \glspl{html} هیچ خروجی ایجاد نخواهد شد.
برای اضافه کردن شکل به خروجی‌های دیگر نیز باید این برچسب را دوباره تکرار کرد. کد
زیر اصلاح شده کد قبل است که نه تنها شکل را در خروجی \glspl{latex} بلکه در خروجی
\glspl{html}‌ نیز اضافه می‌کند.

\begin{Java}
/**
 * Here is a snapshot of my new application:
 *  \image html application.jpg
 *  \image latex application.eps "My application" width=0.5\textwidth
 */
\end{Java}


\begin{warning}
ساختار پرونده شکل مورد استفاده باید به گونه‌ای باشد که در کاوشگرها و یا
مترجم‌های مورد استفاده حمایت شود. برای نمونه استفاده از شکل با ساختار \lr{eps}
در \glspl{html} قابل قبول نیست چرا که توسط کاوشگرها حمایت نمی‌شود.

\glspl{doxygen} ساختار پرونده‌های مورد استفاده در این ورچسب را بررسی نمی‌کند از
این رو کاربران خود موظف به بررسی این موضوع هستند.
\end{warning}


% FIXME: maso 1391: کاراکترهای خاص باید در اینحا تشریح شود.




