%
% حق نشر 1390-1402 دانش پژوهان ققنوس
% حقوق این اثر محفوظ است.
% 
% استفاده مجدد از متن و یا نتایج این اثر در هر شکل غیر قانونی است مگر اینکه متن حق
% نشر بالا در ابتدای تمامی مستندهای و یا برنامه‌های به دست آمده از این اثر
% بازنویسی شود. این کار باید برای تمامی مستندها، متنهای تبلیغاتی برنامه‌های
% کاربردی و سایر مواردی که از این اثر به دست می‌آید مندرج شده و در قسمت تقدیر از
% صاحب این اثر نام برده شود.
% 
% نام گروه دانش پژوهان ققنوس ممکن است در محصولات دست آمده شده از این اثر درج
% نشود که در این حالت با مطالبی که در بالا اورده شده در تضاد نیست. برای اطلاع
% بیشتر در مورد حق نشر آدرس زیر مراجعه کنید:
% 
% http://dpq.co.ir/licence
%
\section{مستند و انواع آن}

در زبان‌های برنامه سازی \glspl{comment} یک ساختار برنامه نویسی است که به عنوان
توضیحات خوانا برای برنامه نویسان در لابه لای کدهای برنامه نوشته می‌شود. گرچه این
توضیحات توسط مترجم‌های زبان برنامه سازی نادیده گرفته می‌شود اما برای توسعه
دهندگان سیستم‌های نرم‌افزاری بسیار حیاطی است. هدف اصلی ساختار \glspl{comment} در
زبان‌های برنامه سازی اسان ساختن درک یک برنامه است اما از آنجا که زبان‌های
برنامه‌سازی از ساختارهای دقیقی در کدها پیروی می‌کنند دور از انتظار نیست که در
این مورد نیز ساختارهای از پیش تعریف شده‌ای ایجاد شود.

گرچه استفاده از \glspl{comment} در ساختارهای برنامه سازی منجر به فهم ساده از
برنامه می‌شود اما استفاده نامناسب از آن و ایجاد اطلاعات نامناسب، کم و یا گاها
اضافی منجر به ناکارآمد شدن آن می‌شود. در این راستا راهکارها و توسیه‌های متفاوتی
از سوی تحلیلگران و توسعه‌دهندگان سیستم‌های نرم‌افزاری ارائه شده است که می‌تواند
در این امر بسیار کارساز و مفید باشد.

به هر حال امروز بسیار مرسوم است که با استفاده از \glspl{comment} مستندهای مفید و
جامعی در مورد سیستم‌ها ایجاد شده و در داخل کد برنامه‌ها قرار گیرد\cite{17wiki}.
برای نمونه \glspl{comment} می‌تواند شامل اطلاعاتی مانند توسعه‌دهنده، نسخه، تاریخ
و یا روش‌های به کار رفته در پیاده‌سازی باشد. بزرگترین خصوصیت استفاده از
ساختار \glspl{comment} در مستند کردن کدهای ایجاد شده، ساده‌تر شدن فرآیند
مستندسازی سیستم‌ها است. از سویی نوشتن مستند در کدهای ایجاد شده این امکان را
ایجاد می‌کند که مستند ایجاد شده همواره به روز باشد. در نهایت مستندهای ایجاد شده
در این ساختارها با استفاده از ابزارهای مستندگر سازماندهی شده به شکلهای مناسب در
اختیار کاربران سیستم قرار خواهد کرد.

% maso 1391: انواع مستندها در این قسمت توضیح داده می‌شود
همواره مستندهای موجود در یک ساختار برنامه‌سازی را می‌توان در دو موضوع کلی
دسته‌بندی کرد: مستند فنی و پیاده‌سازی. در بسیاری از موارد بسته‌های نرم‌افزاری
ایجاد شده به عنوان زیر سیستم‌هایی در سایر سیستم‌ها مورد استفاده قرار می‌گیرند.
در این صورت باید مستندهای کافی در مورد ساختار و نحوه به کارگیری ایجاد شود تا
کاربران (که عموما توسعه دهنده هستند) بتواند به صورت کارا از آن استفاده کنند.

از سویی همواره سیستم‌های نرم‌افزاری نیاز به توسعه و رفع ایراد دارند بدیهی است که
در این حالت می‌بایست اطلاعات کافی در مورد روش‌های پیاده‌سازی و محدودیت‌های آن
ایجاد شود. الگوریتم‌های به کار رفته ساختارهای داده‌ای داخلی و یا مشکلاتی که در
توسعه سیستم غیر قابل پرهیز است باید به صورت کامل تشریح شود تا در صورت تغییر
ساختار تیم توسعه حیات سیستم دچار خدشه نشود. تفاوت اساسی میان این نوع مستند و
مستند فنی مخاطب آن است. این نوع مستند تنها توسط توسعه دهندگان سیستم مورد استفاده
قرار می‌گیرد در حالی که مخاطب مستند فنی کاربران سیستم هستند. دور از انتظار نیست
که در برخی موارد نیاز به ایجاد یک نوع اطلاعات در هر دو نوع مستند باشد.

در حالت کلی نمی‌توان مرزی میان این دو نوع مستند تعیین کرد اما نکته‌ای که در هردو
انها مشترک است، ایجاد انها به دست برنامه نویسان است. ان نوع مستندها چه توسط
توسعه دهندگان مورد استفاده قرار گیرد چه کاربران، می‌بایست توسط تیم توسعه ایجاد
شود. ایجاد مستند فنی بر اساس قواعد و استانداردهایی ایجاد می‌شود تا بتواند مستند
مفیدی را برای کاربران ایجاد کرن درحالی که مستند پیاده‌سازی کاملا بر اساس روابط و
ضوابط درون گروهی است. 
