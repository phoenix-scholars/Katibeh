%
% حق نشر 1390-1402 دانش پژوهان ققنوس
% حقوق این اثر محفوظ است.
% 
% استفاده مجدد از متن و یا نتایج این اثر در هر شکل غیر قانونی است مگر اینکه متن حق
% نشر بالا در ابتدای تمامی مستندهای و یا برنامه‌های به دست آمده از این اثر
% بازنویسی شود. این کار باید برای تمامی مستندها، متنهای تبلیغاتی برنامه‌های
% کاربردی و سایر مواردی که از این اثر به دست می‌آید مندرج شده و در قسمت تقدیر از
% صاحب این اثر نام برده شود.
% 
% نام گروه دانش پژوهان ققنوس ممکن است در محصولات دست آمده شده از این اثر درج
% نشود که در این حالت با مطالبی که در بالا اورده شده در تضاد نیست. برای اطلاع
% بیشتر در مورد حق نشر آدرس زیر مراجعه کنید:
% 
% http://dpq.co.ir/licence
%
% SUGESTION: Hadi - 1391: پیشنهاد می‌کنم عنوان این بخش را «فرمول در داکسی‌زن»
% نامگذاری کنیم. و بخش بعدی آن را فرمول‌نویسی به سبک لاتک
% NOTE: هادی ۶-۱۳۹۱: تغییر عنوان.
% عنوان قبلی این بخش محیط نوشتن فرمول بود.
\section{فرمول‌نویسی در \lr{Doxygen}}
همانطور که در قسمت‌های قبل اشاره شد \lr{Doxygen} از فرمول‌نویسی به سبک
\lr{\LaTeX} پشتیبانی می‌کند. در این قسمت شرح داده می‌شود  که چگونه می‌توان
فرمول‌های به این سبک را در مستندات منبع نوشت. در این قسمت بیان می‌شود که برای
اینکه فرمول‌ها در مستند نهایی تولید شده توسط \lr{Doxygen} به شکلی که می‌خواهیم
ظاهر شوند باید در مستندات منبع چگونه عمل نمود.

در حالت کلی یک فرمول می‌تواند به دو صورت در یک متن قرار داده شود: بین کلمات یک
خط یا در یک خط جداگانه. حالت اول که معمولا فرمول‌هایی ساده دارند به این
صورت است که فرمول مورد نظر بین کلمات یک خط متنی قرار می‌گیرد. در این حالت
فرمول در واقع مثل یک کلمه از آن خط است. در حالت دوم فرمول در خطی
جداگانه قرار می‌گیرد. این روش برای فرمول‌های بزرگ و پیچیده و به خصوص برای
فرمول‌هایی که ممکن است در قسمت‌های دیگر مستند به آن‌ها رجوع شود مناسب است. برای
وارد کردن فرمول در مستندات سه روش وجود دارد (یک روش برای فرمول‌های درون خطی و دو
روش برای فرمول‌های جداگانه) که در ادامه شرح داده می‌شود.

\textbf{درج فرمول بین کلمات یک خط متنی}. برای درج چنین فرمول‌هایی باید
فرمول مورد نظر را بین یک جفت علامت \lr{\textbackslash f\$} قرار داد. در زیر یک
مثال از درج فرمول به این روش آورده شده است:
\begin{latex}
The distance between \f$(x_1,y_1)\f$ and \f$(x_2,y_2)\f$ is
\f$\sqrt{(x_2-x_1)^2+(y_2-y_1)^2}\f$.
\end{latex}
نتیجه نهایی این مثال در مستندات نهایی تولید شده توسط \lr{Doxygen} به صورت زیر
خواهد بود. توجه کنید که چگونه فرمول بین کلمات یک خط قرار می‌گیرد.
\begin{latin}
The distance between $(x_1,y_1)$ and $(x_2,y_2)$ is $\sqrt{(x_2-x_1)^2+(y_2-y_1)^2}$.
\end{latin}

\textbf{درج فرمول در خط جداگانه}. برای درج چنین فرمول‌هایی باید فرمول
مورد نظر بین علامت‌های \lr{\textbackslash f[} و \lr{\textbackslash f]} نوشته شود. فرمول‌هایی
که به این صورت نوشته شوند در مستندات نهایی در خطی جداگانه و در وسط آن خط قرار
داده می‌شوند. در زیر مثالی از درج فرمول به این صورت مشاهده می‌شود:
\begin{latex}
\f[
	|I_2|=\left| \int_{0}^T \psi(t)
    	\left\{
        	u(a,t)-
            	\int_{\gamma(t)}^a
                \frac{d\theta}{k(\theta,t)}
                \int_{a}^\theta c(\xi)u_t(\xi,t)\,d\xi
            \right\} dt
        \right|
\f]
\end{latex}
نتیجه به این صورت خواهد بود:
\[
	|I_2|=\left| \int_{0}^T \psi(t)
		\left\{
			u(a,t)-
				\int_{\gamma(t)}^a
                \frac{d\theta}{k(\theta,t)}
                \int_{a}^\theta c(\xi)u_t(\xi,t)\,d\xi
            \right\} dt
        \right|
\]

\textbf{استفاده از سایر محیط‌های \lr{\LaTeX}} . روش دوم در واقع معادل با
استفاده از محیط \lr{displaymath} در \lr{\LaTeX} است (این محیط، محیط استاندارد \lr{\LaTeX} برای
فرمول‌نویسی است). روش دیگر، استفاده از سایر محیط‌های فرمول‌نویسی است. برای
فرمول‌نویسی در مستندات منبع می‌توان از محیط‌های دیگر \lr{\LaTeX} هم استفاده
نمود. به عنوان مثال محیط‌هایی مثل \lr{equation}، \lr{eqnarray}، \lr{multline}،
\lr{gather}، \lr{flalign} و \lr{alignat} (برای توضیح در مورد این محیط‌ها به قسمت
~\ref{sec:latex-formula} مراجعه کنید). به طور کلی برای درج فرمول با استفاده از
محیطی خاص، باید فرمول مورد نظر بین علامت \lr{\textbackslash 
f[\textit{<environment>}]} که آغاز محیط را نشان می‌دهد و علامت
\lr{\textbackslash f]} که نشان‌دهنده پایان محیط مورد نظر است، قرار گیرد. به جای
\lr{\textit{<environment>}} باید نام محیط مورد نظر ذکر گردد. به عنوان مثال فرض
کنید می‌خواهیم فرمولی در مستندات خود درج کنیم که این فرمول در مستند نهایی از
امکانات محیط \lr{eqnarray} استفاده کند. به عبارتی فرمول را در این محیط بنویسیم.
مثال زیر چنین کاری انجام می‌دهد:
\begin{latex}
	\f{eqnarray*}{
    	g &=& \frac{Gm_2}{r^2} \\
        	&=& \frac{(6.673 \times 10^{-11}\,\mbox{m}^3\,\mbox{kg}^{-1}\,
            	\mbox{s}^{-2})(5.9736 \times
            10^{24}\,\mbox{kg})}{(6371.01\,\mbox{km})^2} \\
          	&=& 9.82066032\,\mbox{m/s}^2
  	\f}
\end{latex}
که در مستندات نهایی به صورت زیر نمایش می‌یابد:
\begin{latin}
	\begin{eqnarray*}
    	g &=& \frac{Gm_2}{r^2} \\
      		&=& \frac{(6.673 \times 10^{-11}\,\mbox{m}^3\,\mbox{kg}^{-1}\,
          		\mbox{s}^{-2})(5.9736 \times
          		10^{24}\,\mbox{kg})}{(6371.01\,\mbox{km})^2} \\
      		&=& 9.82066032\,\mbox{m/s}^2
    \end{eqnarray*}
\end{latin}

% NOTE: هادی ۶-۱۳۹۱: تغییر متن
% قبلا این قسمت از متن رو به صورت آیتم‌هایی در یک لیست آورده بودم و که بعد به نظرم
% رسید مناسب نیست و آن‌ها را ویرایش کرده و به صورت پاراگرافهای جدا نوشتم.

% \begin{enumerate}
%   \item درج فرمول بین کلمات یک خط متنی. برای درج چنین فرمول‌هایی باید فرمول مورد
%   نظر را بین یک جفت علامت \lr{\textbackslash f\$} قرار داد. در زیر یک مثال
%   از درج فرمول به این روش آورده شده است:  
%   \begin{Tex}
%   The distance between \f$(x_1,y_1)\f$ and \f$(x_2,y_2)\f$ is
%   \f$\sqrt{(x_2-x_1)^2+(y_2-y_1)^2}\f$.
%   \end{Tex}
%   نتیجه نهایی این مثال در مستندات نهایی تولید شده توسط \lr{Doxygen} به صورت زیر
%   خواهد بود. توجه کنید که چگونه فرمول بین کلمات یک خط قرار می‌گیرد.
%   \begin{latin}
%   The distance between $(x_1,y_1)$ and $(x_2,y_2)$ is $\sqrt{(x_2-x_1)^2+(y_2-y_1)^2}$.
%   \end{latin}
%   
%   \item درج فرمول در خط جداگانه. برای درج چنین فرمول‌هایی باید فرمول
%   مورد نظر بین علامت‌های \lr{\textbackslash f[} و \lr{\textbackslash f]} نوشته
%   شود. فرمول‌هایی که به این صورت نوشته شوند در مستندات نهایی در خطی جداگانه و در
%   وسط آن خط قرار داده می‌شوند. در زیر مثالی از درج فرمول به این صورت مشاهده می‌شود:  
%   \begin{Tex}
%   \f[
%     |I_2|=\left| \int_{0}^T \psi(t)
%              \left\{
%                 u(a,t)-
%                 \int_{\gamma(t)}^a
%                 \frac{d\theta}{k(\theta,t)}
%                 \int_{a}^\theta c(\xi)u_t(\xi,t)\,d\xi
%              \right\} dt
%           \right|
%   \f]
%   \end{Tex}
%   نتیجه به این صورت خواهد بود:  
%   \[
%     |I_2|=\left| \int_{0}^T \psi(t)
%              \left\{
%                 u(a,t)-
%                 \int_{\gamma(t)}^a
%                 \frac{d\theta}{k(\theta,t)}
%                 \int_{a}^\theta c(\xi)u_t(\xi,t)\,d\xi
%              \right\} dt
%           \right|
%   \]
%    
%   \item استفاده از سایر محیط‌های \lr{\LaTeX}. روش دوم در واقع معادل با
%   استفاده از محیط \lr{displaymath} در \lr{\LaTeX} است (این محیط، محیط
%   استاندارد \lr{\LaTeX} برای فرمول‌نویسی است). روش دیگر، استفاده از سایر
%   محیط‌های فرمول‌نویسی است. برای فرمول‌نویسی در مستندات
%   منبع می‌توان از محیط‌های دیگر \lr{\LaTeX} هم استفاده نمود. به عنوان مثال
%   محیط‌هایی مثل \lr{equation}، \lr{eqnarray}، \lr{multline}، \lr{gather}،
%   \lr{flalign} و \lr{alignat} (برای توضیح در مورد این محیط‌ها به قسمت
%   ~\ref{sec:latex-formula} مراجعه کنید). به طور کلی برای درج فرمول با استفاده از
%   محیطی خاص، باید فرمول مورد نظر بین علامت \lr{\textbackslash
%   f[\textit{<environment>}]} که آغاز محیط را نشان می‌دهد و علامت
%   \lr{\textbackslash f]} که نشان‌دهنده پایان محیط مورد نظر است، قرار
%   گیرد. به جای \lr{\textit{<environment>}} باید نام محیط مورد نظر ذکر گردد. به
%   عنوان مثال فرض کنید می‌خواهیم فرمولی در مستندات خود درج کنیم که این فرمول
%   در مستند نهایی از امکانات محیط \lr{eqnarray} استفاده کند. به عبارتی فرمول را
%   در این محیط بنویسیم. مثال زیر چنین کاری انجام می‌دهد:
%   \begin{Tex}
%   \f{eqnarray*}{
%         g &=& \frac{Gm_2}{r^2} \\
%           &=& \frac{(6.673 \times 10^{-11}\,\mbox{m}^3\,\mbox{kg}^{-1}\,
%               \mbox{s}^{-2})(5.9736 \times 10^{24}\,\mbox{kg})}{(6371.01\,\mbox{km})^2} \\
%           &=& 9.82066032\,\mbox{m/s}^2
%   \f}
%   \end{Tex}
%   که در مستندات نهایی به صورت زیر نمایش می‌یابد:
%   \begin{latin}
%   \begin{eqnarray*}
%     g &=& \frac{Gm_2}{r^2} \\
%       &=& \frac{(6.673 \times 10^{-11}\,\mbox{m}^3\,\mbox{kg}^{-1}\,
%           \mbox{s}^{-2})(5.9736 \times 10^{24}\,\mbox{kg})}{(6371.01\,\mbox{km})^2} \\
%       &=& 9.82066032\,\mbox{m/s}^2
%   \end{eqnarray*}
%   \end{latin}
% 
% \end{enumerate}

% FIXME : مصطفی ۱۳۹۰-۱۲ : بررسی نوشتن مستند به صورت مجزا برای نوشتن مستند به
% صورت مجزا محیط‌های متفاوتی وجود دارد که در اینجا مورد بررسی قرار نگرفته است.
% بهتر است آنها رو به صورت کامل بررسی کنیم.
% RE-FIXME: هادی ۶-۱۳۹۱: من فصل فرمول‌ها رو سه قسمت کردم. قسمت اول درباره
% قالب‌های خروجی قسمت دوم درباره دستورات لاتک برای فرمول‌نویسی و قسمت سوم درباره فرمول‌نویسی به
% سبک لاتک در داکسی‌ژن است (یعنی همین قسمت). محیط‌های مختلف را در قسمت دوم شرح
% خواهیم داد. به همین دلیل در اینجا آورده نشده است.

\begin{note}
برای دو روش اول باید مطمئن باشید که فرمول‌ها با استفاده از دستورات استاندارد
\lr{\LaTeX} برای فرمول‌نویسی نوشته شده‌اند. برای روش سوم هم دستورات به کار رفته
در فرمول‌ها باید برای محیطی که دربرگیرنده فرمول است معتبر باشند.
\end{note}
