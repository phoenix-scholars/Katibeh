
\section{دسته بندی اعضا}

هرگاه یک موجودیت، که خود شامل مستندهای متفاوتی است (برای نمونه یک کلاس را فرض
کنید که شامل روندها، و متغیرهای متفاوت با مستندهای متفاوت است)، پسندیده است که
مستندهای موجود در این موجودیت به صورت مناسب دسته بندی شده باشد. گرچه
\lr{Doxygen} به صورت خودکار اعضای موجود در یک موجودیت را  بر اساس نوع، سطح
دسترسی  و تریب واژکها دسته بندی می‌کند، اما در بسیاری از موارد این گونه
دسته‌بندی‌ها از نظر منطقی کافی نیست. برای نمونه حالتی را تصور کنید که در آن
دسته‌ای از روندها از نظر منطقی با هم در ارتباط هستند، گرچه سطح دسترسی آنها
متفاوت است،  اما قرار داده آنها در یک دسته می‌تواند بسیار مفید باشد. دسته بندی
کردن اعضای یک موجودیت با استفاده از عبارت‌های زیر انجام می‌شود.
\begin{latin}
\lstset{language=C++}  
\begin{lstlisting}[frame=single] 
  /**\{*/
  ...
  /**\}*/
  
  //\{
    ..
  //\}
\end{lstlisting}
\end{latin}

توجه به این نکته ضروری است که، تمام اعضای یک دسته باید به صورت فیزیکی در آن دسته
قرار گیرد. همواره این نیاز  به تعیین یک نام  و توصیف برای یک دسته، در مستند سازی
وجود دارد. برای تعیین کردن نام برای یک دسته از اعضا از برچسب \lr{name} استفاده
می‌شود. ساختار کلی این دستور به صورت زیر است.

\begin{latin}
\lstset{language=C++}  
\begin{lstlisting}[frame=single] 
    /**
     * \name [(header)]
     */
\end{lstlisting}
\end{latin}

که در آن سرایند، نامی است که در مستند تولید شده برای دسته به نمایش در خواهد آمد.
همان گونه که مشخص است تعیین سرایند برای یک دسته از اعضا اختیاری است. مستند مربوط
به دسته بعد از این برچسب نوشته می‌شود به عبارتی، استفاده از این برچسب در یک بسته
مستند به این معنی است که مستند در رابطه با یک دسته از اعضا نوشته شده است.
استفاده از عبارت \lr{ \\{ \\}} بعد از این برچسب اجباری است.

\begin{note}
دسته بندی کردن اعضا به صورت تو در تو مجاز نیست. به بیان دیگر اعضا تنها می‌توانند
در یک سطح دسته بندی شوند.
\end{note}

همواره دسته بندی شدن اعضای یک موجودیت (به ویژه زمانی که این دسته بندی به صورت
خودکار انجام می‌شود) مفید نیست. از این رو برخلاف آنچه که در بالا به آن اشاره شد،
گاهی نیاز است که از دسته بندی شده اعضا جلوگیری کرد. برای نمونه حالتی را تصور
کنید که در آن فهرستی از متغیرهای ایستای عمومی\lr{Public Static} تعریف شده است که
به عنوان کلید و یا مقادیر اولیه در سیستم استفاده می‌شود، اما نوع این متغیرها
متفاوت است. از انجا که نوع این متغیرها متفاوت است \lr{Doxygen} به صورت خودکار
انها را بر اساس نوع در دسته‌های متفاوت قرار خواهد داد. برای جلوگیری کردن از این
حالت از برچسب \lr{nogrouping} استفاده می‌شود. ساختار کلی این دستور به صورت زیر
است.

\begin{latin}
\lstset{language=C++}  
\begin{lstlisting}[frame=single] 
    /**
     * \nogrouping
     */
\end{lstlisting}
\end{latin}

با قرار دادن این برچسب از دسته بندی شدن مستندها به صورت خودکار جلوگیری خواهد شد.
برای روشن شدن چگونگی دسته بندی کردن اعضای یک موجودیت به نمونه زیر توجه کنید.

\begin{latin}
\lstset{language=C++}  
\begin{lstlisting}[frame=single] 
    /** A class. Details */
    class Test{
      public:
	//\{
	/** Same documentation for both members. Details */
	void func1InGroup1();
	void func2InGroup1();
	//\}

	/** Function without group. Details. */
	void ungroupedFunction();
	void func1InGroup2();
      protected:
	void func2InGroup2();
    };
    void Test::func1InGroup1() {}
    void Test::func2InGroup1() {}
    /** \name Group2
    *  Description of group 2. 
    */
    //\{
    /** Function 2 in group 2. Details. */
    void Test::func2InGroup2() {}
    /** Function 1 in group 2. Details. */
    void Test::func1InGroup2() {}
    //\}

    /*! \file 
    *  docs for this file
    */

    //\{
    //! one description for all members of this group 
    //! (because DISTRIBUTE_GROUP_DOC is YES in the config file)
    #define A 1
    #define B 2
    void glob_func();
    //\}
\end{lstlisting}
\end{latin}
در این نمونه \lr{Group1} به عنوان یک زیردسته از \lr{Public Group} تعریف شده است. دسته  \lr{Group2}
به عنوان یک دسته جدا در نظر گرفته خواهد شد، در حالی که روندهای موجود در آن سطح دسترسی متفاوتی را دارند.
