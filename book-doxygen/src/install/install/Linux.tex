% در این قسمت نحوه‌ی نصب، روی سیستم عامل های لینوکسی شرح داده می شود.
% محمد هادی منصوری ۹۰/۳/۲
%
% حق نشر 1390-1402 دانش پژوهان ققنوس
% حقوق این اثر محفوظ است.
% 
% استفاده مجدد از متن و یا نتایج این اثر در هر شکل غیر قانونی است مگر اینکه متن حق
% نشر بالا در ابتدای تمامی مستندهای و یا برنامه‌های به دست آمده از این اثر
% بازنویسی شود. این کار باید برای تمامی مستندها، متنهای تبلیغاتی برنامه‌های
% کاربردی و سایر مواردی که از این اثر به دست می‌آید مندرج شده و در قسمت تقدیر از
% صاحب این اثر نام برده شود.
% 
% نام گروه دانش پژوهان ققنوس ممکن است در محصولات دست آمده شده از این اثر درج
% نشود که در این حالت با مطالبی که در بالا اورده شده در تضاد نیست. برای اطلاع
% بیشتر در مورد حق نشر آدرس زیر مراجعه کنید:
% 
% http://dpq.co.ir/licence
%
\section{لینوکس}
\label{install/linux}

\begin{sloppypar}
برای نصب \lr{Doxygen} روی سیستم عامل‌های لینوکسی نیاز به توزیع دودویی مناسب برای سیستم عامل خود دارید. از آدرس 
\url{http://www.doxygen.org/download.html}
 می‌توانید توزیع مناسب را دریافت کنید. پس از دریافت توزیع مناسب، با دستورات زیر می‌توان \lr{Doxygen} را نصب کرد. 
معمولا بین پرونده‌های دریافت شده، پرونده‌ای به نام \lr{configure} وجود دارد. این پرونده حاوی دستورات خط فرمان (دستورات شل) برای 
پیکربندی اولیه است. دستورات زیر پیکربندی اولیه را انجام داده و برنامه را نصب می‌کند.
\end{sloppypar}
\begin{latin}
\lstset{language=C++}  
\begin{lstlisting}[frame=single] 
./configure
make install
\end{lstlisting}
\end{latin}
%\begin{flushleft}
%\lr{./configure} \\
%\lr{make install}
%\end{flushleft}
برای نصب مستندات و مثال‌ها دستور زیر را اجرا کنید.
\begin{latin}
\lstset{language=C++}  
\begin{lstlisting}[frame=single] 
make install_docs
\end{lstlisting}
\end{latin}
%\begin{flushleft}
%\lr{make install\_docs}
%\end{flushleft}
پرونده‌های اجرایی در مسیر 
\lr{<prefix>/bin}
 و مستندات و مثال‌ها در مسیر 
\lr{<docdir>/doxygen}
  نصب می‌شوند.

\begin{sloppypar}
به صورت پیش‌فرض 
\lr{<prefix>}
 به آدرس 
\lr{usr/local}
 اشاره دارد، اما می‌توان آن را با تغییر پارامتر 
\lr{--prefix}
 در پرونده‌ی \lr{configure} تغییر داد. 
\lr{<docdir>}
 نیز به صورت پیش‌فرض به آدرس 
\lr{<prefix>/share/doc/packages}
 اشاره دارد که این آدرس را نیز می‌توان با تغییر پارامتر 
\lr{--doxdir}
 در پرونده‌ی \lr{configure} عوض کرد.
\end{sloppypar}

در صورتی که از بسته‌های \lr{RPM} یا \lr{DEP} استفاده می‌کنید، همان روند استاندارد 
نصب را دنبال کنید که برای این بسته‌ها مورد نیاز است.

